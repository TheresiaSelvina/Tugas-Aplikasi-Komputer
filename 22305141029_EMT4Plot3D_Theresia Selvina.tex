\documentclass[a4paper,10pt]{article}
\usepackage{eumat}

\begin{document}
\begin{eulernotebook}
\eulerheading{Tugas Aplikasi Komputer}
\begin{eulercomment}
Nama  : Theresia Selvina Vanny Maharani\\
NIM   : 22305141029\\
Kelas : Matematika B

\begin{eulercomment}
\eulerheading{Menggambar Plot 3D dengan EMT}
\begin{eulercomment}
Ini adalah pengantar untuk plot 3D dalam Euler. Kami memerlukan plot
3D untuk memvisualisasikan sebuah fungsi dari dua variabel.

Euler menggambar fungsi-fungsi seperti itu menggunakan algoritma
penyortingan untuk menyembunyikan bagian-bagian di latar belakang.
Secara umum, Euler menggunakan proyeksi sentral. Pengaturan default
adalah dari kuadran x-y positif menuju titik awal x=y=z=0, tetapi
angle=0° melihat ke arah sumbu y. Sudut pandang dan tinggi dapat
diubah.

Euler dapat membuat plot berikut:

- permukaan dengan bayangan dan garis level atau rentang level,\\
- awan titik,\\
- kurva parametrik,\\
- permukaan implisit.

Plot 3D dari sebuah fungsi menggunakan plot3d. Cara termudah adalah
dengan memplot ekspresi dalam x dan y. Parameter r mengatur rentang
plot di sekitar (0,0).
\end{eulercomment}
\begin{eulerprompt}
>aspect(1.5); plot3d("x^2+sin(y)",-5,5,0,6*pi):
\end{eulerprompt}
\eulerimg{17}{images/22305141029_EMT4Plot3D_Theresia Selvina-001.png}
\begin{eulerprompt}
>plot3d("x^2+x*sin(y)",-5,5,0,6*pi):
\end{eulerprompt}
\eulerimg{17}{images/22305141029_EMT4Plot3D_Theresia Selvina-002.png}
\begin{eulercomment}
Silakan lakukan modifikasi agar gambar "talang bergelombang" tersebut tidak lurus melainkan melengkung/melingkar, baik
melingkar secara mendatar maupun melingkar turun/naik (seperti papan peluncur pada kolam renang. Temukan rumusnya.
\end{eulercomment}
\begin{eulerprompt}
>aspect(1.5); plot3d("x^2-y^2+x*sin(y)",-20,20,-6*pi,6*pi):
\end{eulerprompt}
\eulerimg{22}{images/22305141029_EMT4Plot3D_Theresia Selvina-003.png}
\begin{eulerprompt}
>aspect(1.5); plot3d("x^2+y^2-6*x*sin(y)",-20,20,-12*pi,12*pi):
\end{eulerprompt}
\eulerimg{17}{images/22305141029_EMT4Plot3D_Theresia Selvina-004.png}
\eulerheading{Fungsi Dua Variabel}
\begin{eulercomment}
Untuk grafik dari sebuah fungsi, gunakan

- ekspresi sederhana dalam x dan y,\\
- nama fungsi dari dua variabel,\\
- atau matriks data.

Pengaturan default adalah kisi kawat yang diisi dengan warna yang
berbeda di kedua sisi. Perhatikan bahwa jumlah interval kisi default
adalah 10, tetapi plot menggunakan jumlah default 40x40 persegi
panjang untuk membangun permukaan. Ini bisa diubah.

- n=40, n=[40,40]: jumlah garis kisi dalam setiap arah\\
- grid=10, grid=[10,10]: jumlah garis kisi dalam setiap arah.

Kami menggunakan nilai default n=40 dan grid=10.
\end{eulercomment}
\begin{eulerprompt}
>plot3d("x^2+y^2"):
\end{eulerprompt}
\eulerimg{17}{images/22305141029_EMT4Plot3D_Theresia Selvina-005.png}
\begin{eulercomment}
Interaksi pengguna dimungkinkan dengan parameter \textgreater{}user. Pengguna dapat
menekan tombol-tombol berikut:

- left,right,up,down: mengubah sudut pandang tampilan\\
- +,-: zoom in atau zoom out\\
- a: membuat gambar anaglyph (lihat di bawah)\\
- l: beralihkan sumber cahaya (lihat di bawah)\\
- space: mengatur ulang ke pengaturan default\\
- return: mengakhiri interaksi
\end{eulercomment}
\begin{eulerprompt}
>plot3d("exp(-x^2+y^2)",>user, ...
>  title="Turn with the vector keys (press return to finish)"):
\end{eulerprompt}
\eulerimg{17}{images/22305141029_EMT4Plot3D_Theresia Selvina-006.png}
\begin{eulercomment}
Rentang plot untuk fungsi dapat ditentukan dengan

- a, b: rentang x\\
- c, d: rentang y\\
- r: kotak simetris di sekitar (0,0).\\
- n: jumlah subinterval untuk plot.

Ada beberapa parameter untuk menyesuaikan skala fungsi atau mengubah
tampilan grafik.

fscale: mengubah skala nilai fungsi (default adalah \textless{}fscale).\\
scale: angka atau vektor 1x2 untuk mengubah skala ke arah x dan y.\\
frame: jenis bingkai (default 1).
\end{eulercomment}
\begin{eulerprompt}
>plot3d("exp(-(x^2+y^2)/5)",r=10,n=80,fscale=4,scale=1.2,frame=3,>user):
\end{eulerprompt}
\eulerimg{17}{images/22305141029_EMT4Plot3D_Theresia Selvina-007.png}
\begin{eulercomment}
Tampilan dapat diubah dengan berbagai cara yang berbeda.

- distance: jarak pandang ke plot.\\
- zoom: nilai zoom.\\
- angle: sudut terhadap sumbu y negatif dalam radian.\\
- height: tinggi pandangan dalam radian.

Nilai default dapat diperiksa atau diubah dengan fungsi view(). Fungsi
ini mengembalikan parameter dalam urutan yang disebutkan di atas.
\end{eulercomment}
\begin{eulerprompt}
>view
\end{eulerprompt}
\begin{euleroutput}
  [5,  2.6,  2,  0.4]
\end{euleroutput}
\begin{eulercomment}
Jarak pandang yang lebih dekat memerlukan zoom yang lebih sedikit.
Efeknya lebih mirip lensa wide angle.

Pada contoh berikut, angle=0 dan height=0 melihat dari sumbu y
negatif. Label sumbu untuk y disembunyikan dalam kasus ini.
\end{eulercomment}
\begin{eulerprompt}
>plot3d("x^2+y",distance=3,zoom=1,angle=pi/2,height=0):
\end{eulerprompt}
\eulerimg{17}{images/22305141029_EMT4Plot3D_Theresia Selvina-008.png}
\begin{eulercomment}
Plot selalu menghadap ke pusat kubus plot. Anda dapat memindahkan
pusatnya dengan parameter center.
\end{eulercomment}
\begin{eulerprompt}
>plot3d("x^4+y^2",a=0,b=1,c=-1,d=1,angle=-20°,height=20°, ...
>  center=[0.4,0,0],zoom=5):
\end{eulerprompt}
\eulerimg{17}{images/22305141029_EMT4Plot3D_Theresia Selvina-009.png}
\begin{eulercomment}
Plotnya akan diubah skala agar sesuai dalam kubus unit untuk
ditampilkan. Jadi tidak perlu mengubah jarak atau zoom tergantung pada
ukuran plot. Namun, label-label mengacu pada ukuran sebenarnya.

Jika Anda mematikan fitur ini dengan scale=false, Anda perlu
memastikan bahwa plot tetap sesuai dalam jendela plotting, dengan
mengubah jarak pandang atau zoom, dan memindahkan pusatnya.
\end{eulercomment}
\begin{eulerprompt}
>plot3d("5*exp(-x^2-y^2)",r=2,<fscale,<scale,distance=13,height=50°, ...
>  center=[0,0,-2],frame=3):
\end{eulerprompt}
\eulerimg{17}{images/22305141029_EMT4Plot3D_Theresia Selvina-010.png}
\begin{eulercomment}
Tersedia juga plot polar. Parameter polar=true menggambar plot polar.
Fungsi masih harus menjadi fungsi dari x dan y. Parameter "fscale"
mengubah skala fungsi dengan skala sendiri. Selain itu, fungsi akan
diubah skala agar sesuai dalam kubus.
\end{eulercomment}
\begin{eulerprompt}
>plot3d("1/(x^2+y^2+1)",r=5,>polar, ...
>fscale=2,>hue,n=100,zoom=4,>contour,color=blue):
\end{eulerprompt}
\eulerimg{17}{images/22305141029_EMT4Plot3D_Theresia Selvina-011.png}
\begin{eulerprompt}
>function f(r) := exp(-r/2)*cos(r); ...
>plot3d("f(x^2+y^2)",>polar,scale=[1,1,0.4],r=pi,frame=3,zoom=4):
\end{eulerprompt}
\eulerimg{17}{images/22305141029_EMT4Plot3D_Theresia Selvina-012.png}
\begin{eulercomment}
Parameter rotate memutar fungsi dalam x sekitar sumbu x.

- rotate=1: Menggunakan sumbu x\\
- rotate=2: Menggunakan sumbu z
\end{eulercomment}
\begin{eulerprompt}
>plot3d("x^2+1",a=-1,b=1,rotate=true,grid=5):
\end{eulerprompt}
\eulerimg{17}{images/22305141029_EMT4Plot3D_Theresia Selvina-013.png}
\begin{eulerprompt}
>plot3d("x^2+1",a=-1,b=1,rotate=2,grid=5):
\end{eulerprompt}
\eulerimg{17}{images/22305141029_EMT4Plot3D_Theresia Selvina-014.png}
\begin{eulerprompt}
>plot3d("sqrt(25-x^2)",a=0,b=5,rotate=1):
\end{eulerprompt}
\eulerimg{17}{images/22305141029_EMT4Plot3D_Theresia Selvina-015.png}
\begin{eulerprompt}
>plot3d("x*sin(x)",a=0,b=6pi,rotate=2):
\end{eulerprompt}
\eulerimg{17}{images/22305141029_EMT4Plot3D_Theresia Selvina-016.png}
\begin{eulercomment}
Berikut adalah plot dengan tiga fungsi.
\end{eulercomment}
\begin{eulerprompt}
>plot3d("x","x^2+y^2","y",r=2,zoom=3.5,frame=3):
\end{eulerprompt}
\eulerimg{17}{images/22305141029_EMT4Plot3D_Theresia Selvina-017.png}
\eulerheading{Plot Kontur}
\begin{eulercomment}
Untuk plot ini, Euler menambahkan garis-garis kisi. Sebaliknya, Anda
dapat menggunakan garis level dan variasi warna satu warna atau
variasi warna spektral. Euler dapat menggambar ketinggian fungsi pada
plot dengan shading. Dalam semua plot 3D, Euler dapat menghasilkan
anaglyph merah/cyan.

- \textgreater{}hue: Mengaktifkan shading ringan alih-alih kawat.\\
- \textgreater{}contour: Menggambar garis kontur otomatis pada plot.\\
- level=... (atau levels): Sebuah vektor nilai untuk garis kontur.

Defaultnya adalah level="auto", yang menghitung beberapa garis level
secara otomatis. Seperti yang Anda lihat dalam plot, level sebenarnya
adalah rentang level.

Gaya default dapat diubah. Untuk plot kontur berikutnya, kita
menggunakan grid yang lebih halus dengan 100x100 titik, mengubah skala
fungsi dan plot, dan menggunakan sudut pandang yang berbeda.
\end{eulercomment}
\begin{eulerprompt}
>plot3d("exp(-x^2-y^2)",r=2,n=100,level="thin", ...
> >contour,>spectral,fscale=1,scale=1.1,angle=45°,height=20°):
\end{eulerprompt}
\eulerimg{17}{images/22305141029_EMT4Plot3D_Theresia Selvina-018.png}
\begin{eulerprompt}
>plot3d("exp(x*y)",angle=100°,>contour,color=green):
\end{eulerprompt}
\eulerimg{17}{images/22305141029_EMT4Plot3D_Theresia Selvina-019.png}
\begin{eulercomment}
Pengaturan shading default menggunakan warna abu-abu. Namun, tersedia
juga rentang warna spektral.

- \textgreater{}spectral: Menggunakan skema spektral default.\\
- color=...: Menggunakan warna khusus atau skema spektral.

Untuk plot berikutnya, kami menggunakan skema spektral default dan
meningkatkan jumlah titik untuk mendapatkan tampilan yang sangat
halus.
\end{eulercomment}
\begin{eulerprompt}
>plot3d("x^2+y^2",>spectral,>contour,n=100):
\end{eulerprompt}
\eulerimg{17}{images/22305141029_EMT4Plot3D_Theresia Selvina-020.png}
\begin{eulercomment}
Daripada menggunakan garis level otomatis, kita juga dapat mengatur
nilai-nilai garis level. Ini akan menghasilkan garis level tipis
alih-alih rentang level.
\end{eulercomment}
\begin{eulerprompt}
>plot3d("x^2-y^2",0,5,0,5,level=-1:0.1:1,color=redgreen):
\end{eulerprompt}
\eulerimg{17}{images/22305141029_EMT4Plot3D_Theresia Selvina-021.png}
\begin{eulercomment}
Pada plot berikut, kita menggunakan dua pita level yang sangat lebar,
mulai dari -0.1 hingga 1, dan dari 0.9 hingga 1. Ini dimasukkan
sebagai matriks dengan batas level sebagai kolom.

Selain itu, kita melampirkan kisi dengan 10 interval dalam setiap
arah.
\end{eulercomment}
\begin{eulerprompt}
>plot3d("x^2+y^3",level=[-0.1,0.9;0,1], ...
>  >spectral,angle=30°,grid=10,contourcolor=gray):
\end{eulerprompt}
\eulerimg{17}{images/22305141029_EMT4Plot3D_Theresia Selvina-022.png}
\begin{eulercomment}
Pada contoh berikut, kita memplot himpunan, di mana

\end{eulercomment}
\begin{eulerformula}
\[
f(x,y)=x^y-y^x=0
\]
\end{eulerformula}
\begin{eulercomment}
Kita menggunakan satu garis tipis untuk garis level.
\end{eulercomment}
\begin{eulerprompt}
>plot3d("x^y-y^x",level=0,a=0,b=6,c=0,d=6,contourcolor=red,n=100):
\end{eulerprompt}
\eulerimg{17}{images/22305141029_EMT4Plot3D_Theresia Selvina-024.png}
\begin{eulercomment}
Mungkin untuk menampilkan sebuah bidang kontur di bawah plot. Warna
dan jaraknya ke plot dapat ditentukan.
\end{eulercomment}
\begin{eulerprompt}
>plot3d("x^2+y^4",>cp,cpcolor=green,cpdelta=0.2):
\end{eulerprompt}
\eulerimg{17}{images/22305141029_EMT4Plot3D_Theresia Selvina-025.png}
\begin{eulercomment}
Berikut beberapa gaya lainnya. Kami selalu mematikan bingkai dan
menggunakan berbagai skema warna untuk plot dan kisi.
\end{eulercomment}
\begin{eulerprompt}
>figure(2,2); ...
>expr="y^3-x^2"; ...
>figure(1);  ...
>  plot3d(expr,<frame,>cp,cpcolor=spectral); ...
>figure(2);  ...
>  plot3d(expr,<frame,>spectral,grid=10,cp=2); ...
>figure(3);  ...
>  plot3d(expr,<frame,>contour,color=gray,nc=5,cp=3,cpcolor=greenred); ...
>figure(4);  ...
>  plot3d(expr,<frame,>hue,grid=10,>transparent,>cp,cpcolor=gray); ...
>figure(0):
\end{eulerprompt}
\eulerimg{17}{images/22305141029_EMT4Plot3D_Theresia Selvina-026.png}
\begin{eulercomment}
Ada beberapa skema spektral lainnya, diberi nomor dari 1 hingga 9.
Tetapi Anda juga dapat menggunakan color=value, di mana value

- spectral: untuk rentang dari biru hingga merah\\
- white: untuk rentang yang lebih redup\\
- yellowblue, purplegreen, blueyellow, greenred\\
- blueyellow, greenpurple, yellowblue, redgreen
\end{eulercomment}
\begin{eulerprompt}
>figure(3,3); ...
>for i=1:9;  ...
>  figure(i); plot3d("x^2+y^2",spectral=i,>contour,>cp,<frame,zoom=4);  ...
>end; ...
>figure(0):
\end{eulerprompt}
\eulerimg{17}{images/22305141029_EMT4Plot3D_Theresia Selvina-027.png}
\begin{eulercomment}
Sumber cahaya dapat diubah dengan l dan tombol-tombol kursor selama
interaksi pengguna. Ini juga dapat diatur dengan parameter-parameter
berikut:

- light: arah cahaya\\
- amb: cahaya ambien antara 0 dan 1

Perlu diperhatikan bahwa program ini tidak membuat perbedaan antara
sisi-sisi plot. Tidak ada bayangan. Untuk itu, Anda akan memerlukan
Povray.
\end{eulercomment}
\begin{eulerprompt}
>plot3d("-x^2-y^2", ...
>  hue=true,light=[0,1,1],amb=0,user=true, ...
>  title="Press l and cursor keys (return to exit)"):
\end{eulerprompt}
\eulerimg{17}{images/22305141029_EMT4Plot3D_Theresia Selvina-028.png}
\begin{eulercomment}
Parameter warna mengubah warna permukaan. Warna garis level juga dapat
diubah.
\end{eulercomment}
\begin{eulerprompt}
>plot3d("-x^2-y^2",color=rgb(0.2,0.2,0),hue=true,frame=false, ...
>  zoom=3,contourcolor=red,level=-2:0.1:1,dl=0.01):
\end{eulerprompt}
\eulerimg{17}{images/22305141029_EMT4Plot3D_Theresia Selvina-029.png}
\begin{eulercomment}
Warna 0 memberikan efek pelangi khusus.
\end{eulercomment}
\begin{eulerprompt}
>plot3d("x^2/(x^2+y^2+1)",color=0,hue=true,grid=10):
\end{eulerprompt}
\eulerimg{17}{images/22305141029_EMT4Plot3D_Theresia Selvina-030.png}
\begin{eulercomment}
Permukaan juga bisa transparan.
\end{eulercomment}
\begin{eulerprompt}
>plot3d("x^2+y^2",>transparent,grid=10,wirecolor=red):
\end{eulerprompt}
\eulerimg{17}{images/22305141029_EMT4Plot3D_Theresia Selvina-031.png}
\eulerheading{Plot Implisit}
\begin{eulercomment}
Terdapat juga plot implisit dalam tiga dimensi. Euler menghasilkan
potongan-potongan melalui objek. Fitur-fitur dari plot3d termasuk plot
implisit. Plot ini menampilkan himpunan nol dari sebuah fungsi dalam
tiga variabel.\\
Solusi dari

\end{eulercomment}
\begin{eulerformula}
\[
f(x,y,z) = 0
\]
\end{eulerformula}
\begin{eulercomment}
Ini bisa divisualisasikan dalam potongan yang sejajar dengan bidang
x-y, x-z, dan y-z.

- implicit=1: potongan sejajar dengan bidang y-z\\
- implicit=2: potongan sejajar dengan bidang x-z\\
- implicit=4: potongan sejajar dengan bidang x-y

Tambahkan nilai-nilai ini jika Anda suka. Pada contoh di atas, kita
memplot

\end{eulercomment}
\begin{eulerformula}
\[
M = \{ (x,y,z) : x^2+y^3+zy=1 \}
\]
\end{eulerformula}
\begin{eulerprompt}
>plot3d("x^2+y^3+z*y-1",r=5,implicit=3):
\end{eulerprompt}
\eulerimg{17}{images/22305141029_EMT4Plot3D_Theresia Selvina-034.png}
\begin{eulerprompt}
>c=1; d=1;
>plot3d("((x^2+y^2-c^2)^2+(z^2-1)^2)*((y^2+z^2-c^2)^2+(x^2-1)^2)*((z^2+x^2-c^2)^2+(y^2-1)^2)-d",r=2,<frame,>implicit,>user): 
\end{eulerprompt}
\eulerimg{17}{images/22305141029_EMT4Plot3D_Theresia Selvina-035.png}
\begin{eulerprompt}
>plot3d("x^2+y^2+4*x*z+z^3",>implicit,r=2,zoom=2.5):
\end{eulerprompt}
\eulerimg{17}{images/22305141029_EMT4Plot3D_Theresia Selvina-036.png}
\eulerheading{Plotting 3D Data}
\begin{eulercomment}
Sama seperti plot2d, plot3d juga menerima data. Untuk objek 3D, Anda
perlu menyediakan matriks nilai x, y, dan z, atau tiga fungsi atau
ekspresi fx(x, y), fy(x, y), fz(x, y).

\end{eulercomment}
\begin{eulerformula}
\[
\gamma(t,s) = (x(t,s),y(t,s),z(t,s))
\]
\end{eulerformula}
\begin{eulercomment}
Karena x, y, z adalah matriks, kami mengasumsikan bahwa (t, s)
berjalan melalui grid persegi. Akibatnya, Anda dapat memplot
gambar-gambar persegi panjang di dalam ruang.

Anda dapat menggunakan bahasa matriks Euler untuk menghasilkan
koordinat dengan efektif.

Pada contoh berikut, kami menggunakan vektor nilai t dan vektor kolom
nilai s untuk memarameterkan permukaan bola. Dalam gambar, kita dapat
menandai wilayah-wilayah, dalam hal ini wilayah polar.
\end{eulercomment}
\begin{eulerprompt}
>t=linspace(0,2pi,180); s=linspace(-pi/2,pi/2,90)'; ...
>x=cos(s)*cos(t); y=cos(s)*sin(t); z=sin(s); ...
>plot3d(x,y,z,>hue, ...
>color=blue,<frame,grid=[10,20], ...
>values=s,contourcolor=red,level=[90°-24°;90°-22°], ...
>scale=1.4,height=50°):
\end{eulerprompt}
\eulerimg{17}{images/22305141029_EMT4Plot3D_Theresia Selvina-038.png}
\begin{eulercomment}
Berikut adalah contoh, yang merupakan grafik dari sebuah fungsi.
\end{eulercomment}
\begin{eulerprompt}
>t=-1:0.1:1; s=(-1:0.1:1)'; plot3d(t,s,t*s,grid=10):
\end{eulerprompt}
\eulerimg{17}{images/22305141029_EMT4Plot3D_Theresia Selvina-039.png}
\begin{eulercomment}
Namun, kita bisa membuat berbagai macam permukaan. Berikut adalah
permukaan yang sama sebagai fungsi

\end{eulercomment}
\begin{eulerformula}
\[
x = y \, z
\]
\end{eulerformula}
\begin{eulerprompt}
>plot3d(t*s,t,s,angle=180°,grid=10):
\end{eulerprompt}
\eulerimg{17}{images/22305141029_EMT4Plot3D_Theresia Selvina-041.png}
\begin{eulercomment}
Dengan usaha lebih, kita bisa menghasilkan banyak permukaan yang
berbeda.

Pada contoh berikutnya, kita membuat tampilan berbayang dari bola yang
distorsi. Koordinat biasa untuk bola tersebut adalah

\end{eulercomment}
\begin{eulerformula}
\[
\gamma(t,s) = (\cos(t)\cos(s),\sin(t)\sin(s),\cos(s))
\]
\end{eulerformula}
\begin{eulercomment}
dengan

\end{eulercomment}
\begin{eulerformula}
\[
0 \le t \le 2\pi, \quad \frac{-\pi}{2} \le s \le \frac{\pi}{2}.
\]
\end{eulerformula}
\begin{eulercomment}
Kita merubahnya dengan faktor.

\end{eulercomment}
\begin{eulerformula}
\[
d(t,s) = \frac{\cos(4t)+\cos(8s)}{4}.
\]
\end{eulerformula}
\begin{eulerprompt}
>t=linspace(0,2pi,320); s=linspace(-pi/2,pi/2,160)'; ...
>d=1+0.2*(cos(4*t)+cos(8*s)); ...
>plot3d(cos(t)*cos(s)*d,sin(t)*cos(s)*d,sin(s)*d,hue=1, ...
>  light=[1,0,1],frame=0,zoom=5):
\end{eulerprompt}
\eulerimg{17}{images/22305141029_EMT4Plot3D_Theresia Selvina-045.png}
\begin{eulercomment}
Tentu saja, titik-titik data juga dimungkinkan. Untuk memplot data
titik di dalam ruang, kita memerlukan tiga vektor untuk koordinat
titik-titik tersebut.

Gaya-gaya yang digunakan sama seperti dalam plot2d dengan points=true;
\end{eulercomment}
\begin{eulerprompt}
>n=500;  ...
>  plot3d(normal(1,n),normal(1,n),normal(1,n),points=true,style="."):
\end{eulerprompt}
\eulerimg{17}{images/22305141029_EMT4Plot3D_Theresia Selvina-046.png}
\begin{eulercomment}
Juga mungkin untuk memplot kurva dalam 3D. Dalam kasus ini, lebih
mudah untuk menghitung sebelumnya titik-titik kurva. Untuk kurva dalam
bidang, kita menggunakan urutan koordinat dan parameter wire=true.
\end{eulercomment}
\begin{eulerprompt}
>t=linspace(0,8pi,500); ...
>plot3d(sin(t),cos(t),t/10,>wire,zoom=3):
\end{eulerprompt}
\eulerimg{17}{images/22305141029_EMT4Plot3D_Theresia Selvina-047.png}
\begin{eulerprompt}
>t=linspace(0,4pi,1000); plot3d(cos(t),sin(t),t/2pi,>wire, ...
>linewidth=3,wirecolor=blue):
\end{eulerprompt}
\eulerimg{17}{images/22305141029_EMT4Plot3D_Theresia Selvina-048.png}
\begin{eulerprompt}
>X=cumsum(normal(3,100)); ...
> plot3d(X[1],X[2],X[3],>anaglyph,>wire):
\end{eulerprompt}
\eulerimg{17}{images/22305141029_EMT4Plot3D_Theresia Selvina-049.png}
\begin{eulercomment}
EMT juga dapat membuat plot dalam mode anaglyph. Untuk melihat plot
seperti ini, Anda memerlukan kacamata merah/cyan.
\end{eulercomment}
\begin{eulerprompt}
> plot3d("x^2+y^3",>anaglyph,>contour,angle=30°):
\end{eulerprompt}
\eulerimg{17}{images/22305141029_EMT4Plot3D_Theresia Selvina-050.png}
\begin{eulercomment}
Seringkali, skema warna spektral digunakan untuk plot. Ini menekankan
tinggi fungsi.
\end{eulercomment}
\begin{eulerprompt}
>plot3d("x^2*y^3-y",>spectral,>contour,zoom=3.2):
\end{eulerprompt}
\eulerimg{17}{images/22305141029_EMT4Plot3D_Theresia Selvina-051.png}
\begin{eulercomment}
Euler juga dapat memplot permukaan yang diparameterisasi, ketika
parameter-parameter tersebut adalah nilai-nilai x, y, dan z dari
gambar grid persegi panjang di dalam ruang.

Untuk demo berikutnya, kami menyiapkan parameter u dan v, dan
menghasilkan koordinat ruang dari kedua parameter tersebut.
\end{eulercomment}
\begin{eulerprompt}
>u=linspace(-1,1,10); v=linspace(0,2*pi,50)'; ...
>X=(3+u*cos(v/2))*cos(v); Y=(3+u*cos(v/2))*sin(v); Z=u*sin(v/2); ...
>plot3d(X,Y,Z,>anaglyph,<frame,>wire,scale=2.3):
\end{eulerprompt}
\eulerimg{17}{images/22305141029_EMT4Plot3D_Theresia Selvina-052.png}
\begin{eulercomment}
Berikut adalah contoh yang lebih rumit, yang menjadi megah saat
dilihat dengan kacamata merah/cyan.
\end{eulercomment}
\begin{eulerprompt}
>u:=linspace(-pi,pi,160); v:=linspace(-pi,pi,400)';  ...
>x:=(4*(1+.25*sin(3*v))+cos(u))*cos(2*v); ...
>y:=(4*(1+.25*sin(3*v))+cos(u))*sin(2*v); ...
> z=sin(u)+2*cos(3*v); ...
>plot3d(x,y,z,frame=0,scale=1.5,hue=1,light=[1,0,-1],zoom=2.8,>anaglyph):
\end{eulerprompt}
\eulerimg{17}{images/22305141029_EMT4Plot3D_Theresia Selvina-053.png}
\eulerheading{Plot Statistik}
\begin{eulercomment}
Plot batang juga dimungkinkan. Untuk ini, kita harus menyediakan:

- x: vektor baris dengan n+1 elemen\\
- y: vektor kolom dengan n+1 elemen\\
- z: matriks nxn dari nilai-nilai.

z bisa lebih besar, tetapi hanya nilai-nilai nxn yang akan digunakan.

Pada contoh di atas, kita pertama-tama menghitung nilai-nilai.
Kemudian kita menyesuaikan x dan y, sehingga vektor-vektor tersebut
berpusat pada nilai yang digunakan.
\end{eulercomment}
\begin{eulerprompt}
>x=-1:0.1:1; y=x'; z=x^2+y^2; ...
>xa=(x|1.1)-0.05; ya=(y_1.1)-0.05; ...
>plot3d(xa,ya,z,bar=true):
\end{eulerprompt}
\eulerimg{17}{images/22305141029_EMT4Plot3D_Theresia Selvina-054.png}
\begin{eulercomment}
Anda dapat membagi plot dari sebuah permukaan menjadi dua atau lebih
bagian.
\end{eulercomment}
\begin{eulerprompt}
>x=-1:0.1:1; y=x'; z=x+y; d=zeros(size(x)); ...
>plot3d(x,y,z,disconnect=2:2:20):
\end{eulerprompt}
\eulerimg{17}{images/22305141029_EMT4Plot3D_Theresia Selvina-055.png}
\begin{eulercomment}
Jika Anda memuat atau menghasilkan matriks data M dari sebuah file dan
perlu memplotnya dalam 3D, Anda bisa mengubah skala matriks ke [-1,1]
dengan scale(M), atau mengubah skala matriks dengan \textgreater{}zscale. Ini dapat
digabungkan dengan faktor skala individu yang diterapkan secara
tambahan.
\end{eulercomment}
\begin{eulerprompt}
>i=1:20; j=i'; ...
>plot3d(i*j^2+100*normal(20,20),>zscale,scale=[1,1,1.5],angle=-40°,zoom=1.8):
\end{eulerprompt}
\eulerimg{17}{images/22305141029_EMT4Plot3D_Theresia Selvina-056.png}
\begin{eulerprompt}
>Z=intrandom(5,100,6); v=zeros(5,6); ...
>loop 1 to 5; v[#]=getmultiplicities(1:6,Z[#]); end; ...
>columnsplot3d(v',scols=1:5,ccols=[1:5]):
\end{eulerprompt}
\eulerimg{17}{images/22305141029_EMT4Plot3D_Theresia Selvina-057.png}
\eulerheading{Permukaan Benda Putar}
\begin{eulerprompt}
>plot2d("(x^2+y^2-1)^3-x^2*y^3",r=1.3, ...
>style="#",color=red,<outline, ...
>level=[-2;0],n=100):
\end{eulerprompt}
\eulerimg{17}{images/22305141029_EMT4Plot3D_Theresia Selvina-058.png}
\begin{eulerprompt}
>ekspresi &= (x^2+y^2-1)^3-x^2*y^3; $ekspresi
\end{eulerprompt}
\begin{eulerformula}
\[
\left(y^2+x^2-1\right)^3-x^2\,y^3
\]
\end{eulerformula}
\begin{eulercomment}
Kita ingin memutar kurva hati sekitar sumbu y. Berikut adalah ekspresi
yang mendefinisikan kurva hati:

\end{eulercomment}
\begin{eulerformula}
\[
f(x,y)=(x^2+y^2-1)^3-x^2.y^3.
\]
\end{eulerformula}
\begin{eulercomment}
Selanjutnya, kita menetapkan

\end{eulercomment}
\begin{eulerformula}
\[
x=r.cos(a),\quad y=r.sin(a).
\]
\end{eulerformula}
\begin{eulerprompt}
>function fr(r,a) &= ekspresi with [x=r*cos(a),y=r*sin(a)] | trigreduce; $fr(r,a)
\end{eulerprompt}
\begin{eulerformula}
\[
\left(r^2-1\right)^3+\frac{\left(\sin \left(5\,a\right)-\sin \left(  3\,a\right)-2\,\sin a\right)\,r^5}{16}
\]
\end{eulerformula}
\begin{eulercomment}
Ini memungkinkan untuk mendefinisikan sebuah fungsi numerik, yang akan
menyelesaikan untuk r jika a sudah diberikan. Dengan fungsi itu, kita
bisa memplot hati yang telah berputar sebagai permukaan parametrik.
\end{eulercomment}
\begin{eulerprompt}
>function map f(a) := bisect("fr",0,2;a); ...
>t=linspace(-pi/2,pi/2,100); r=f(t);  ...
>s=linspace(pi,2pi,100)'; ...
>plot3d(r*cos(t)*sin(s),r*cos(t)*cos(s),r*sin(t), ...
>>hue,<frame,color=red,zoom=4,amb=0,max=0.7,grid=12,height=50°):
\end{eulerprompt}
\eulerimg{17}{images/22305141029_EMT4Plot3D_Theresia Selvina-063.png}
\begin{eulercomment}
Berikut adalah plot 3D dari gambar di atas yang diputar sekitar sumbu
z. Kita mendefinisikan fungsi yang menggambarkan objek tersebut.
\end{eulercomment}
\begin{eulerprompt}
>function f(x,y,z) ...
\end{eulerprompt}
\begin{eulerudf}
  r=x^2+y^2;
  return (r+z^2-1)^3-r*z^3;
   endfunction
\end{eulerudf}
\begin{eulerprompt}
>plot3d("f(x,y,z)", ...
>xmin=0,xmax=1.2,ymin=-1.2,ymax=1.2,zmin=-1.2,zmax=1.4, ...
>implicit=1,angle=-30°,zoom=2.5,n=[10,100,60],>anaglyph):
\end{eulerprompt}
\eulerimg{17}{images/22305141029_EMT4Plot3D_Theresia Selvina-064.png}
\eulerheading{Special 3D Plots}
\begin{eulercomment}
Fungsi plot3d bagus untuk dimiliki, tetapi tidak selalu memenuhi semua
kebutuhan. Selain rutinitas yang lebih dasar, mungkin juga
memungkinkan untuk mendapatkan plot bingkai dari objek apa pun yang
Anda suka.

Meskipun Euler bukanlah program 3D, ia dapat menggabungkan beberapa
objek dasar. Kami mencoba untuk memvisualisasikan sebuah paraboloid
dan garis singgungnya.
\end{eulercomment}
\begin{eulerprompt}
>function myplot ...
\end{eulerprompt}
\begin{eulerudf}
    y=-1:0.01:1; x=(-1:0.01:1)';
    plot3d(x,y,0.2*(x-0.1)/2,<scale,<frame,>hue, ..
      hues=0.5,>contour,color=orange);
    h=holding(1);
    plot3d(x,y,(x^2+y^2)/2,<scale,<frame,>contour,>hue);
    holding(h);
  endfunction
\end{eulerudf}
\begin{eulercomment}
Sekarang framedplot() memberikan bingkai-bingkai tersebut, dan
mengatur pandangannya.
\end{eulercomment}
\begin{eulerprompt}
>framedplot("myplot",[-1,1,-1,1,0,1],height=0,angle=-30°, ...
>  center=[0,0,-0.7],zoom=3):
\end{eulerprompt}
\eulerimg{17}{images/22305141029_EMT4Plot3D_Theresia Selvina-065.png}
\begin{eulercomment}
Dengan cara yang sama, Anda dapat memplot bidang kontur secara manual.
Perlu diingat bahwa plot3d() secara default mengatur jendela ke
fullwindow(), tetapi plotcontourplane() mengasumsikan hal itu.
\end{eulercomment}
\begin{eulerprompt}
>x=-1:0.02:1.1; y=x'; z=x^2-y^4;
>function myplot (x,y,z) ...
\end{eulerprompt}
\begin{eulerudf}
    zoom(2);
    wi=fullwindow();
    plotcontourplane(x,y,z,level="auto",<scale);
    plot3d(x,y,z,>hue,<scale,>add,color=white,level="thin");
    window(wi);
    reset();
  endfunction
\end{eulerudf}
\begin{eulerprompt}
>myplot(x,y,z):
\end{eulerprompt}
\eulerimg{27}{images/22305141029_EMT4Plot3D_Theresia Selvina-066.png}
\eulerheading{Animasi}
\begin{eulercomment}
Euler dapat menggunakan bingkai untuk menghitung animasi sebelumnya.

Salah satu fungsi yang menggunakan teknik ini adalah rotate. Ini dapat
mengubah sudut pandang dan menggambar ulang plot 3D. Fungsi ini
memanggil addpage() untuk setiap plot baru. Akhirnya, ia
menganimasikan plot-plot tersebut.

Silakan pelajari sumber kode rotate untuk melihat lebih banyak
detailnya.
\end{eulercomment}
\begin{eulerprompt}
>function testplot () := plot3d("x^2+y^3"); ...
>rotate("testplot"); testplot():
\end{eulerprompt}
\eulerimg{27}{images/22305141029_EMT4Plot3D_Theresia Selvina-067.png}
\eulerheading{Menggambar Povray}
\begin{eulercomment}
Dengan bantuan berkas Euler povray.e, Euler dapat menghasilkan berkas
Povray. Hasilnya sangat bagus untuk dilihat.

Anda perlu menginstal Povray (32bit atau 64bit) dari
http://www.povray.org/ , dan menempatkan sub-direktori "bin" dari Povray ke dalam path lingkungan, atau mengatur variabel "defaultpovray" dengan path lengkap yang mengarah ke "pvengine.exe".

Antarmuka Povray Euler menghasilkan berkas Povray di direktori home
pengguna, dan memanggil Povray untuk mengurai berkas-berkas ini. Nama
berkas default adalah current.pov, dan direktori default adalah
eulerhome(), biasanya c:\textbackslash{}Users\textbackslash{}Username\textbackslash{}Euler. Povray menghasilkan
berkas PNG, yang dapat dimuat oleh Euler ke dalam notebook. Untuk
membersihkan berkas-berkas ini, gunakan povclear().

Fungsi pov3d memiliki semangat yang sama dengan plot3d. Ini dapat
menghasilkan grafik dari fungsi f(x,y), atau permukaan dengan
koordinat X, Y, Z dalam matriks, termasuk garis level opsional. Fungsi
ini secara otomatis memulai raytracer, dan memuat adegan ke dalam
notebook Euler.

Selain pov3d(), ada banyak fungsi yang menghasilkan objek Povray.
Fungsi-fungsi ini mengembalikan string yang berisi kode Povray untuk
objek-objek tersebut. Untuk menggunakan fungsi-fungsi ini, mulai
berkas Povray dengan povstart(). Kemudian gunakan writeln(...) untuk
menulis objek-objek ke berkas adegan. Terakhir, akhiri berkas dengan
povend(). Secara default, raytracer akan mulai, dan PNG akan
dimasukkan ke dalam notebook Euler.

Fungsi objek memiliki parameter yang disebut "look", yang memerlukan
string dengan kode Povray untuk tekstur dan finishing objek. Fungsi
povlook() dapat digunakan untuk menghasilkan string ini. Ini memiliki
parameter untuk warna, transparansi, Phong Shading, dll.

Perhatikan bahwa alam semesta Povray memiliki sistem koordinat lain.
Antarmuka ini menerjemahkan semua koordinat ke sistem Povray. Jadi
Anda bisa terus berpikir dalam sistem koordinat Euler dengan z
mengarah vertikal ke atas, dan sumbu x, y, z searah dengan tangan
kanan. Anda perlu memuat berkas povray.
\end{eulercomment}
\begin{eulerprompt}
>load povray;
\end{eulerprompt}
\begin{eulercomment}
Pastikan direktori bin Povray berada dalam path. Jika tidak, edit
variabel berikut sehingga berisi path ke eksekutor povray.
\end{eulercomment}
\begin{eulerprompt}
>defaultpovray="C:\(\backslash\)Program Files\(\backslash\)POV-Ray\(\backslash\)v3.7\(\backslash\)bin\(\backslash\)pvengine.exe"
\end{eulerprompt}
\begin{euleroutput}
  C:\(\backslash\)Program Files\(\backslash\)POV-Ray\(\backslash\)v3.7\(\backslash\)bin\(\backslash\)pvengine.exe
\end{euleroutput}
\begin{eulercomment}
Untuk memberikan kesan pertama, kita memplot sebuah fungsi sederhana.
Perintah berikut menghasilkan berkas povray di direktori pengguna
Anda, dan menjalankan Povray untuk ray tracing berkas ini.

Jika Anda menjalankan perintah berikut, GUI Povray seharusnya terbuka,
menjalankan berkas, dan menutup secara otomatis. Karena alasan
keamanan, Anda akan ditanya apakah Anda ingin mengizinkan file exe
untuk berjalan. Anda dapat menekan "cancel" untuk menghentikan
pertanyaan lebih lanjut. Anda mungkin perlu menekan OK di jendela
Povray untuk mengakui dialog startup Povray.
\end{eulercomment}
\begin{eulerprompt}
>plot3d("x^2+y^2",zoom=2):
\end{eulerprompt}
\eulerimg{22}{images/22305141029_EMT4Plot3D_Theresia Selvina-068.png}
\begin{eulerprompt}
>pov3d("x^2+y^2",zoom=3);
\end{eulerprompt}
\eulerimg{27}{images/22305141029_EMT4Plot3D_Theresia Selvina-069.png}
\begin{eulercomment}
Kita dapat membuat fungsi tersebut menjadi transparan dan menambahkan
finishing lainnya. Kita juga dapat menambahkan garis level ke plot
fungsi tersebut.
\end{eulercomment}
\begin{eulerprompt}
>pov3d("x^2+y^3",axiscolor=red,angle=-45°,>anaglyph, ...
>  look=povlook(cyan,0.2),level=-1:0.5:1,zoom=3.8);
\end{eulerprompt}
\eulerimg{27}{images/22305141029_EMT4Plot3D_Theresia Selvina-070.png}
\begin{eulercomment}
Terkadang perlu untuk mencegah penskalaan fungsi dan mengubah skala
fungsi secara manual. 

Kita memplot himpunan titik di bidang kompleks, di mana hasil kali
jarak ke 1 dan -1 sama dengan 1.
\end{eulercomment}
\begin{eulerprompt}
>pov3d("((x-1)^2+y^2)*((x+1)^2+y^2)/40",r=2, ...
>  angle=-120°,level=1/40,dlevel=0.005,light=[-1,1,1],height=10°,n=50, ...
>  <fscale,zoom=3.8);
\end{eulerprompt}
\eulerimg{27}{images/22305141029_EMT4Plot3D_Theresia Selvina-071.png}
\eulerheading{Memplot dengan koordinat}
\begin{eulercomment}
Alih-alih menggunakan fungsi, kita bisa memplot dengan koordinat.
Seperti pada plot3d, kita memerlukan tiga matriks untuk mendefinisikan
objek tersebut.

Pada contoh ini, kita memutar sebuah fungsi sekitar sumbu z.
\end{eulercomment}
\begin{eulerprompt}
>function f(x) := x^3-x+1; ...
>x=-1:0.01:1; t=linspace(0,2pi,50)'; ...
>Z=x; X=cos(t)*f(x); Y=sin(t)*f(x); ...
>pov3d(X,Y,Z,angle=40°,look=povlook(red,0.1),height=50°,axis=0,zoom=4,light=[10,5,15]);
\end{eulerprompt}
\eulerimg{27}{images/22305141029_EMT4Plot3D_Theresia Selvina-072.png}
\begin{eulercomment}
Pada contoh berikut, kita memplot gelombang yang meredam. Kita
menghasilkan gelombang tersebut dengan bahasa matriks Euler.

Kita juga menunjukkan bagaimana objek tambahan dapat ditambahkan ke
dalam sebuah adegan pov3d. Untuk menghasilkan objek, lihat
contoh-contoh berikutnya. Perhatikan bahwa plot3d mengubah skala plot
sehingga sesuai dalam kubus satuan.
\end{eulercomment}
\begin{eulerprompt}
>r=linspace(0,1,80); phi=linspace(0,2pi,80)'; ...
>x=r*cos(phi); y=r*sin(phi); z=exp(-5*r)*cos(8*pi*r)/3;  ...
>pov3d(x,y,z,zoom=6,axis=0,height=30°,add=povsphere([0.5,0,0.25],0.15,povlook(red)), ...
>  w=500,h=300);
\end{eulerprompt}
\eulerimg{18}{images/22305141029_EMT4Plot3D_Theresia Selvina-073.png}
\begin{eulercomment}
Dengan metode shading Povray yang canggih, sangat sedikit titik dapat
menghasilkan permukaan yang sangat halus. Hanya di batas dan dalam
bayangan mungkin trik ini menjadi jelas.

Untuk ini, kita perlu menambahkan vektor normal di setiap titik
matriks.
\end{eulercomment}
\begin{eulerprompt}
>Z &= x^2*y^3
\end{eulerprompt}
\begin{euleroutput}
  
                                   2  3
                                  x  y
  
\end{euleroutput}
\begin{eulercomment}
Persamaan permukaannya adalah [x, y, Z]. Kita menghitung dua turunan
terhadap x dan y dari ini dan mengambil hasil perkalian silangnya
sebagai vektor normal.
\end{eulercomment}
\begin{eulerprompt}
>dx &= diff([x,y,Z],x); dy &= diff([x,y,Z],y);
\end{eulerprompt}
\begin{eulercomment}
Kita mendefinisikan vektor normal sebagai hasil perkalian silang dari
turunan-turunan ini, dan mendefinisikan fungsi koordinat.
\end{eulercomment}
\begin{eulerprompt}
>N &= crossproduct(dx,dy); NX &= N[1]; NY &= N[2]; NZ &= N[3]; N,
\end{eulerprompt}
\begin{euleroutput}
  
                                 3       2  2
                         [- 2 x y , - 3 x  y , 1]
  
\end{euleroutput}
\begin{eulercomment}
Kita hanya menggunakan 25 poin.
\end{eulercomment}
\begin{eulerprompt}
>x=-1:0.5:1; y=x';
>pov3d(x,y,Z(x,y),angle=10°, ...
>  xv=NX(x,y),yv=NY(x,y),zv=NZ(x,y),<shadow);
\end{eulerprompt}
\eulerimg{27}{images/22305141029_EMT4Plot3D_Theresia Selvina-074.png}
\begin{eulercomment}
Berikut adalah simpul Trefoil yang dibuat oleh A. Busser dalam Povray.
Ada versi yang diperbarui dari ini dalam contoh-contoh.

See: Examples\textbackslash{}Trefoil Knot \textbar{} Trefoil Knot

Untuk tampilan yang baik dengan tidak terlalu banyak titik, kita
menambahkan vektor normal di sini. Kami menggunakan Maxima untuk
menghitung vektor normal untuk kita. Pertama, tiga fungsi untuk
koordinat sebagai ekspresi simbolik.
\end{eulercomment}
\begin{eulerprompt}
>X &= ((4+sin(3*y))+cos(x))*cos(2*y); ...
>Y &= ((4+sin(3*y))+cos(x))*sin(2*y); ...
>Z &= sin(x)+2*cos(3*y);
\end{eulerprompt}
\begin{eulercomment}
Kemudian vektor turunan kedua terhadap x dan y.
\end{eulercomment}
\begin{eulerprompt}
>dx &= diff([X,Y,Z],x); dy &= diff([X,Y,Z],y);
\end{eulerprompt}
\begin{eulercomment}
Sekarang vektor normal, yang merupakan hasil perkalian silang dari
kedua turunan tersebut.
\end{eulercomment}
\begin{eulerprompt}
>dn &= crossproduct(dx,dy);
\end{eulerprompt}
\begin{eulercomment}
Sekarang kita mengevaluasi semua ini secara numerik.
\end{eulercomment}
\begin{eulerprompt}
>x:=linspace(-%pi,%pi,40); y:=linspace(-%pi,%pi,100)';
\end{eulerprompt}
\begin{eulercomment}
Vektor normal adalah hasil evaluasi dari ekspresi simbolik dn[i] untuk
i=1,2,3. Sintaks untuk ini adalah \&"ekspresi"(parameter). Ini adalah
alternatif dari metode dalam contoh sebelumnya, di mana kita
mendefinisikan ekspresi simbolik NX, NY, NZ terlebih dahulu.
\end{eulercomment}
\begin{eulerprompt}
>pov3d(X(x,y),Y(x,y),Z(x,y),>anaglyph,axis=0,zoom=5,w=450,h=350, ...
>  <shadow,look=povlook(blue), ...
>  xv=&"dn[1]"(x,y), yv=&"dn[2]"(x,y), zv=&"dn[3]"(x,y));
\end{eulerprompt}
\eulerimg{21}{images/22305141029_EMT4Plot3D_Theresia Selvina-075.png}
\begin{eulercomment}
Kita juga bisa menghasilkan grid dalam 3D.
\end{eulercomment}
\begin{eulerprompt}
>povstart(zoom=4); ...
>x=-1:0.5:1; r=1-(x+1)^2/6; ...
>t=(0°:30°:360°)'; y=r*cos(t); z=r*sin(t); ...
>writeln(povgrid(x,y,z,d=0.02,dballs=0.05)); ...
>povend();
\end{eulerprompt}
\eulerimg{27}{images/22305141029_EMT4Plot3D_Theresia Selvina-076.png}
\begin{eulercomment}
Dengan povgrid(), kurva-kurva juga mungkin.
\end{eulercomment}
\begin{eulerprompt}
>povstart(center=[0,0,1],zoom=3.6); ...
>t=linspace(0,2,1000); r=exp(-t); ...
>x=cos(2*pi*10*t)*r; y=sin(2*pi*10*t)*r; z=t; ...
>writeln(povgrid(x,y,z,povlook(red))); ...
>writeAxis(0,2,axis=3); ...
>povend();
\end{eulerprompt}
\eulerimg{27}{images/22305141029_EMT4Plot3D_Theresia Selvina-077.png}
\eulerheading{Povray Objects}
\begin{eulercomment}
Di atas, kami menggunakan pov3d untuk memplot permukaan. Antarmuka
povray dalam Euler juga dapat menghasilkan objek Povray. Objek-objek
ini disimpan sebagai string dalam Euler, dan perlu ditulis ke berkas
Povray.

Kami memulai output dengan povstart().
\end{eulercomment}
\begin{eulerprompt}
>povstart(zoom=4);
\end{eulerprompt}
\begin{eulercomment}
Pertama, kita mendefinisikan tiga silinder, dan menyimpannya dalam
string-string dalam Euler.

Fungsi-fungsi povx() dll. hanya mengembalikan vektor [1,0,0], yang
bisa digunakan sebagai gantinya.
\end{eulercomment}
\begin{eulerprompt}
>c1=povcylinder(-povx,povx,1,povlook(red)); ...
>c2=povcylinder(-povy,povy,1,povlook(yellow)); ...
>c3=povcylinder(-povz,povz,1,povlook(blue)); ...
\end{eulerprompt}
\begin{eulercomment}
String-string tersebut berisi kode Povray, yang tidak perlu kita
pahami pada saat itu.
\end{eulercomment}
\begin{eulerprompt}
>c2
\end{eulerprompt}
\begin{euleroutput}
  cylinder \{ <0,0,-1>, <0,0,1>, 1
   texture \{ pigment \{ color rgb <0.941176,0.941176,0.392157> \}  \} 
   finish \{ ambient 0.2 \} 
   \}
\end{euleroutput}
\begin{eulercomment}
Seperti yang Anda lihat, kami menambahkan tekstur ke objek dalam tiga
warna yang berbeda.

Hal itu dilakukan oleh povlook(), yang mengembalikan sebuah string
dengan kode Povray yang relevan. Kita dapat menggunakan warna-warna
default Euler, atau mendefinisikan warna sendiri. Kita juga dapat
menambahkan transparansi, atau mengubah cahaya ambien.
\end{eulercomment}
\begin{eulerprompt}
>povlook(rgb(0.1,0.2,0.3),0.1,0.5)
\end{eulerprompt}
\begin{euleroutput}
   texture \{ pigment \{ color rgbf <0.101961,0.2,0.301961,0.1> \}  \} 
   finish \{ ambient 0.5 \} 
  
\end{euleroutput}
\begin{eulercomment}
Sekarang kita mendefinisikan sebuah objek perpotongan, dan menulis
hasilnya ke dalam berkas.
\end{eulercomment}
\begin{eulerprompt}
>writeln(povintersection([c1,c2,c3]));
\end{eulerprompt}
\begin{eulercomment}
Perpotongan dari tiga silinder sulit untuk divisualisasikan, jika Anda
belum pernah melihatnya sebelumnya.
\end{eulercomment}
\begin{eulerprompt}
>povend;
\end{eulerprompt}
\eulerimg{27}{images/22305141029_EMT4Plot3D_Theresia Selvina-078.png}
\begin{eulercomment}
Fungsi-fungsi berikut menghasilkan fraktal secara rekursif.

Fungsi pertama menunjukkan bagaimana Euler menangani objek Povray
sederhana. Fungsi povbox() mengembalikan string yang berisi koordinat
kotak, tekstur, dan finishingnya.
\end{eulercomment}
\begin{eulerprompt}
>function onebox(x,y,z,d) := povbox([x,y,z],[x+d,y+d,z+d],povlook());
>function fractal (x,y,z,h,n) ...
\end{eulerprompt}
\begin{eulerudf}
   if n==1 then writeln(onebox(x,y,z,h));
   else
     h=h/3;
     fractal(x,y,z,h,n-1);
     fractal(x+2*h,y,z,h,n-1);
     fractal(x,y+2*h,z,h,n-1);
     fractal(x,y,z+2*h,h,n-1);
     fractal(x+2*h,y+2*h,z,h,n-1);
     fractal(x+2*h,y,z+2*h,h,n-1);
     fractal(x,y+2*h,z+2*h,h,n-1);
     fractal(x+2*h,y+2*h,z+2*h,h,n-1);
     fractal(x+h,y+h,z+h,h,n-1);
   endif;
  endfunction
\end{eulerudf}
\begin{eulerprompt}
>povstart(fade=10,<shadow);
>fractal(-1,-1,-1,2,4);
>povend();
\end{eulerprompt}
\eulerimg{27}{images/22305141029_EMT4Plot3D_Theresia Selvina-079.png}
\begin{eulercomment}
Perbedaan memungkinkan untuk memotong satu objek dari objek lainnya.
Seperti perpotongan, ini adalah bagian dari objek CSG dari Povray.
\end{eulercomment}
\begin{eulerprompt}
>povstart(light=[5,-5,5],fade=10);
\end{eulerprompt}
\begin{eulercomment}
Untuk demonstrasi ini, kita mendefinisikan sebuah objek dalam Povray,
alih-alih menggunakan string dalam Euler. Definisi diteruskan ke
berkas secara langsung.

Koordinat kotak -1 hanya berarti [-1,-1,-1].
\end{eulercomment}
\begin{eulerprompt}
>povdefine("mycube",povbox(-1,1));
\end{eulerprompt}
\begin{eulercomment}
Kita dapat menggunakan objek ini dalam povobject(), yang mengembalikan
sebuah string seperti biasanya.
\end{eulercomment}
\begin{eulerprompt}
>c1=povobject("mycube",povlook(red));
\end{eulerprompt}
\begin{eulercomment}
Kita menghasilkan sebuah kubus kedua, dan memutar serta mengubah
skalanya sedikit.
\end{eulercomment}
\begin{eulerprompt}
>c2=povobject("mycube",povlook(yellow),translate=[1,1,1], ...
>  rotate=xrotate(10°)+yrotate(10°), scale=1.2);
\end{eulerprompt}
\begin{eulercomment}
Kemudian kita mengambil perbedaan dari dua objek tersebut.
\end{eulercomment}
\begin{eulerprompt}
>writeln(povdifference(c1,c2));
\end{eulerprompt}
\begin{eulercomment}
Sekarang tambahkan tiga sumbu.
\end{eulercomment}
\begin{eulerprompt}
>writeAxis(-1.2,1.2,axis=1); ...
>writeAxis(-1.2,1.2,axis=2); ...
>writeAxis(-1.2,1.2,axis=4); ...
>povend();
\end{eulerprompt}
\eulerimg{27}{images/22305141029_EMT4Plot3D_Theresia Selvina-080.png}
\eulerheading{Fungsi Implisit}
\begin{eulercomment}
Povray dapat memplot himpunan di mana f(x, y, z) = 0, sama seperti
parameter implisit dalam plot3d. Hasilnya terlihat jauh lebih baik,
namun sintaks untuk fungsi-fungsi ini sedikit berbeda. Anda tidak
dapat menggunakan output dari Maxima atau ekspresi Euler.

\end{eulercomment}
\begin{eulerformula}
\[
((x^2+y^2-c^2)^2+(z^2-1)^2)*((y^2+z^2-c^2)^2+(x^2-1)^2)*((z^2+x^2-c^2)^2+(y^2-1)^2)=d
\]
\end{eulerformula}
\begin{eulerprompt}
>povstart(angle=70°,height=50°,zoom=4);
>c=0.1; d=0.1; ...
>writeln(povsurface("(pow(pow(x,2)+pow(y,2)-pow(c,2),2)+pow(pow(z,2)-1,2))*(pow(pow(y,2)+pow(z,2)-pow(c,2),2)+pow(pow(x,2)-1,2))*(pow(pow(z,2)+pow(x,2)-pow(c,2),2)+pow(pow(y,2)-1,2))-d",povlook(red))); ...
>povend();
\end{eulerprompt}
\begin{euleroutput}
  Error : Povray error!
  
  Error generated by error() command
  
  povray:
      error("Povray error!");
  Try "trace errors" to inspect local variables after errors.
  povend:
      povray(file,w,h,aspect,exit); 
\end{euleroutput}
\begin{eulerprompt}
>povstart(angle=25°,height=10°); 
>writeln(povsurface("pow(x,2)+pow(y,2)*pow(z,2)-1",povlook(blue),povbox(-2,2,"")));
>povend();
\end{eulerprompt}
\eulerimg{27}{images/22305141029_EMT4Plot3D_Theresia Selvina-082.png}
\begin{eulerprompt}
>povstart(angle=70°,height=50°,zoom=4);
\end{eulerprompt}
\begin{eulercomment}
Buat permukaan implisit. Perhatikan sintaks yang berbeda dalam
ekspresinya.
\end{eulercomment}
\begin{eulerprompt}
>writeln(povsurface("pow(x,2)*y-pow(y,3)-pow(z,2)",povlook(green))); ...
>writeAxes(); ...
>povend();
\end{eulerprompt}
\eulerimg{27}{images/22305141029_EMT4Plot3D_Theresia Selvina-083.png}
\eulerheading{Mesh Object (Objek Jaringan)}
\begin{eulercomment}
Dalam contoh ini, kami akan menunjukkan bagaimana membuat objek
jaringan (mesh object), dan menggambarnya dengan informasi tambahan.

Kami ingin memaksimalkan xy dengan kondisi x+y=1 dan menunjukkan
sentuhan tangensial dari garis level.
\end{eulercomment}
\begin{eulerprompt}
>povstart(angle=-10°,center=[0.5,0.5,0.5],zoom=7);
\end{eulerprompt}
\begin{eulercomment}
Kita tidak dapat menyimpan objek dalam sebuah string seperti
sebelumnya, karena terlalu besar. Jadi kita mendefinisikan objek dalam
sebuah berkas Povray menggunakan #declare. Fungsi povtriangle()
melakukan ini secara otomatis. Ini dapat menerima vektor normal
seperti pov3d().

Berikut mendefinisikan objek jaringan (mesh object), dan menuliskannya
langsung ke dalam berkas.
\end{eulercomment}
\begin{eulerprompt}
>x=0:0.02:1; y=x'; z=x*y; vx=-y; vy=-x; vz=1;
>mesh=povtriangles(x,y,z,"",vx,vy,vz);
\end{eulerprompt}
\begin{eulercomment}
Sekarang kita mendefinisikan dua cakram (disc), yang akan dipotong
dengan permukaan tersebut.
\end{eulercomment}
\begin{eulerprompt}
>cl=povdisc([0.5,0.5,0],[1,1,0],2); ...
>ll=povdisc([0,0,1/4],[0,0,1],2);
\end{eulerprompt}
\begin{eulercomment}
Tulis permukaan dikurangi dua cakram tersebut.
\end{eulercomment}
\begin{eulerprompt}
>writeln(povdifference(mesh,povunion([cl,ll]),povlook(green)));
\end{eulerprompt}
\begin{eulercomment}
Tulis dua potongan hasil interseksi.
\end{eulercomment}
\begin{eulerprompt}
>writeln(povintersection([mesh,cl],povlook(red))); ...
>writeln(povintersection([mesh,ll],povlook(gray)));
\end{eulerprompt}
\begin{eulercomment}
Tulis sebuah titik pada nilai maksimum.
\end{eulercomment}
\begin{eulerprompt}
>writeln(povpoint([1/2,1/2,1/4],povlook(gray),size=2*defaultpointsize));
\end{eulerprompt}
\begin{eulercomment}
Tambahkan sumbu-sumbu dan selesaikan.
\end{eulercomment}
\begin{eulerprompt}
>writeAxes(0,1,0,1,0,1,d=0.015); ...
>povend();
\end{eulerprompt}
\eulerimg{27}{images/22305141029_EMT4Plot3D_Theresia Selvina-084.png}
\eulerheading{Anaglyphs in Povray}
\begin{eulercomment}
Untuk menghasilkan anaglyph untuk kacamata merah/cyan, Povray harus
dijalankan dua kali dari posisi kamera yang berbeda. Ini menghasilkan
dua berkas Povray dan dua berkas PNG, yang dimuat dengan fungsi
loadanaglyph().

Tentu saja, Anda memerlukan kacamata merah/cyan untuk melihat
contoh-contoh berikut dengan baik.

Fungsi pov3d() memiliki sakelar sederhana untuk menghasilkan anaglyph.
\end{eulercomment}
\begin{eulerprompt}
>pov3d("-exp(-x^2-y^2)/2",r=2,height=45°,>anaglyph, ...
>  center=[0,0,0.5],zoom=3.5);
\end{eulerprompt}
\eulerimg{27}{images/22305141029_EMT4Plot3D_Theresia Selvina-085.png}
\begin{eulercomment}
Jika Anda membuat sebuah adegan dengan objek-objek, Anda perlu
menempatkan pembuatan adegan tersebut dalam sebuah fungsi, dan
menjalankannya dua kali dengan nilai yang berbeda untuk parameter
anaglyph.
\end{eulercomment}
\begin{eulerprompt}
>function myscene ...
\end{eulerprompt}
\begin{eulerudf}
    s=povsphere(povc,1);
    cl=povcylinder(-povz,povz,0.5);
    clx=povobject(cl,rotate=xrotate(90°));
    cly=povobject(cl,rotate=yrotate(90°));
    c=povbox([-1,-1,0],1);
    un=povunion([cl,clx,cly,c]);
    obj=povdifference(s,un,povlook(red));
    writeln(obj);
    writeAxes();
  endfunction
\end{eulerudf}
\begin{eulercomment}
Fungsi povanaglyph() melakukan semua ini. Parameter-parameter ini
seperti dalam povstart() dan povend() yang digabungkan.
\end{eulercomment}
\begin{eulerprompt}
>povanaglyph("myscene",zoom=4.5);
\end{eulerprompt}
\eulerimg{27}{images/22305141029_EMT4Plot3D_Theresia Selvina-086.png}
\eulerheading{Mendefinisikan Objek Sendiri}
\begin{eulercomment}
Antarmuka povray Euler berisi banyak objek. Tetapi Anda tidak terbatas
pada objek-objek ini. Anda dapat membuat objek-objek sendiri, yang
menggabungkan objek-objek lain, atau objek-objek yang benar-benar
baru.

Kami akan menunjukkan sebuah torus sebagai contoh. Perintah Povray
untuk ini adalah "torus". Jadi kita mengembalikan sebuah string dengan
perintah ini dan parameter-parameternya. Perhatikan bahwa torus selalu
berpusat di titik asal.
\end{eulercomment}
\begin{eulerprompt}
>function povdonat (r1,r2,look="") ...
\end{eulerprompt}
\begin{eulerudf}
    return "torus \{"+r1+","+r2+look+"\}";
  endfunction
\end{eulerudf}
\begin{eulercomment}
Inilah torus pertama kita.
\end{eulercomment}
\begin{eulerprompt}
>t1=povdonat(0.8,0.2)
\end{eulerprompt}
\begin{euleroutput}
  torus \{0.8,0.2\}
\end{euleroutput}
\begin{eulercomment}
Mari kita gunakan objek ini untuk membuat torus kedua, yang
diterjemahkan dan diputar.
\end{eulercomment}
\begin{eulerprompt}
>t2=povobject(t1,rotate=xrotate(90°),translate=[0.8,0,0])
\end{eulerprompt}
\begin{euleroutput}
  object \{ torus \{0.8,0.2\}
   rotate 90 *x 
   translate <0.8,0,0>
   \}
\end{euleroutput}
\begin{eulercomment}
Sekarang kita tempatkan objek-objek ini ke dalam sebuah adegan. Untuk
penampilannya, kita gunakan Phong Shading.
\end{eulercomment}
\begin{eulerprompt}
>povstart(center=[0.4,0,0],angle=0°,zoom=3.8,aspect=1.5); ...
>writeln(povobject(t1,povlook(green,phong=1))); ...
>writeln(povobject(t2,povlook(green,phong=1))); ...
\end{eulerprompt}
\begin{eulerttcomment}
 >povend();
\end{eulerttcomment}
\begin{eulercomment}
Program ini memanggil program Povray. Namun, jika terjadi kesalahan,
program ini tidak menampilkan pesan kesalahan. Oleh karena itu, Anda
sebaiknya menggunakan

\end{eulercomment}
\begin{eulerttcomment}
 >povend(<exit);
\end{eulerttcomment}
\begin{eulercomment}

Jika ada sesuatu yang tidak berfungsi, ini akan membuat jendela Povray
tetap terbuka.
\end{eulercomment}
\begin{eulerprompt}
>povend(h=320,w=480);
\end{eulerprompt}
\eulerimg{18}{images/22305141029_EMT4Plot3D_Theresia Selvina-087.png}
\begin{eulercomment}
Berikut contoh yang lebih rinci. Kita selesaikan

\end{eulercomment}
\begin{eulerformula}
\[
Ax \le b, \quad x \ge 0, \quad c.x \to \text{Max.}
\]
\end{eulerformula}
\begin{eulercomment}
dan menampilkan titik-titik layak dan nilai optimum dalam sebuah plot
3D.
\end{eulercomment}
\begin{eulerprompt}
>A=[10,8,4;5,6,8;6,3,2;9,5,6];
>b=[10,10,10,10]';
>c=[1,1,1];
\end{eulerprompt}
\begin{eulercomment}
Pertama, mari kita periksa, apakah contoh ini memiliki solusi.
\end{eulercomment}
\begin{eulerprompt}
>x=simplex(A,b,c,>max,>check)'
\end{eulerprompt}
\begin{euleroutput}
  [0,  1,  0.5]
\end{euleroutput}
\begin{eulercomment}
Ya, itu memiliki solusi.

Kemudian, kita definisikan dua objek. Yang pertama adalah bidang.

\end{eulercomment}
\begin{eulerformula}
\[
a \cdot x \le b
\]
\end{eulerformula}
\begin{eulerprompt}
>function oneplane (a,b,look="") ...
\end{eulerprompt}
\begin{eulerudf}
    return povplane(a,b,look)
  endfunction
\end{eulerudf}
\begin{eulercomment}
Kemudian, kita definisikan perpotongan setengah ruang dan kubus.
\end{eulercomment}
\begin{eulerprompt}
>function adm (A, b, r, look="") ...
\end{eulerprompt}
\begin{eulerudf}
    ol=[];
    loop 1 to rows(A); ol=ol|oneplane(A[#],b[#]); end;
    ol=ol|povbox([0,0,0],[r,r,r]);
    return povintersection(ol,look);
  endfunction
\end{eulerudf}
\begin{eulercomment}
sekarang kita dapat memplot untuk ini.
\end{eulercomment}
\begin{eulerprompt}
>povstart(angle=120°,center=[0.5,0.5,0.5],zoom=3.5); ...
>writeln(adm(A,b,2,povlook(green,0.4))); ...
>writeAxes(0,1.3,0,1.6,0,1.5); ...
\end{eulerprompt}
\begin{eulercomment}
Berikut adalah gambar lingkaran di sekitar nilai optimum.
\end{eulercomment}
\begin{eulerprompt}
>writeln(povintersection([povsphere(x,0.5),povplane(c,c.x')], ...
>  povlook(red,0.9)));
\end{eulerprompt}
\begin{eulercomment}
Dan kesalahan dalam arah optimum.
\end{eulercomment}
\begin{eulerprompt}
>writeln(povarrow(x,c*0.5,povlook(red)));
\end{eulerprompt}
\begin{eulercomment}
Kita tambahkan teks ke layar. Teks ini hanyalah sebuah objek 3D. Kita
perlu menempatkan dan memutarnya sesuai dengan tampilan kita.
\end{eulercomment}
\begin{eulerprompt}
>writeln(povtext("Linear Problem",[0,0.2,1.3],size=0.05,rotate=5°)); ...
>povend();
\end{eulerprompt}
\eulerimg{27}{images/22305141029_EMT4Plot3D_Theresia Selvina-090.png}
\eulerheading{More Examples}
\begin{eulercomment}
Anda dapat menemukan beberapa contoh lainnya untuk Povray di Euler
dalam file-file berikut

See: Examples/Dandelin Spheres\\
See: Examples/Donat Math\\
See: Examples/Trefoil Knot\\
See: Examples/Optimization by Affine Scaling


\begin{eulercomment}
\eulerheading{Soal - soal}
\begin{eulercomment}
1. Sketsakan grafik\\
\end{eulercomment}
\begin{eulerformula}
\[
4x^2+9y^2=36
\]
\end{eulerformula}
\begin{eulercomment}
Penyelesaian :

\end{eulercomment}
\begin{eulerprompt}
>plot3d("4*x^2+9*y^2-36",r=5, implicit=1):
\end{eulerprompt}
\eulerimg{17}{images/22305141029_EMT4Plot3D_Theresia Selvina-092.png}
\begin{eulercomment}
Pada penyelesaian diatas, kita selesaikan dalam bentuk ekspresi
langsung dan menggunakan implicit=1 yang berarti memotong sejajar
dengan sumbu y-z


2. Buatlah sketsa grafik permukaan linear

\end{eulercomment}
\begin{eulerformula}
\[
f(x,y)=x-y-7
\]
\end{eulerformula}
\begin{eulerprompt}
>function f(x,y):= x-y-7
>plot3d("f"):
\end{eulerprompt}
\eulerimg{17}{images/22305141029_EMT4Plot3D_Theresia Selvina-094.png}
\begin{eulercomment}
Pada penyelesaian diatas, digunakan variabel ekspresi untuk menyimpan
rumus fungsi yang akan digambar grafiknya.


3. Gambarkan kurva perpotongan dua persamaan berikut

\end{eulercomment}
\begin{eulerformula}
\[
x^2+y^2+z-4=0
\]
\end{eulerformula}
\begin{eulerformula}
\[
y=1
\]
\end{eulerformula}
\begin{eulerprompt}
>plot3d("x^2+y^2+z-4", r=5, implicit=3);
>plot3d("x^2+y^2-1",>add, implicit=1, r=5):
\end{eulerprompt}
\eulerimg{34}{images/22305141029_EMT4Plot3D_Theresia Selvina-097.png}
\begin{eulercomment}
Kita dapat menggambarkan dua buah grafik dalam satu ruang dengan
menggunakan fugsi \textgreater{}add.


4. Gambarkan

\end{eulercomment}
\begin{eulerformula}
\[
f(x,y)=e^{-(x^2+y^2)}
\]
\end{eulerformula}
\begin{eulerprompt}
>plot3d("exp(-(x^2+y^2))",r=4):
\end{eulerprompt}
\eulerimg{27}{images/22305141029_EMT4Plot3D_Theresia Selvina-099.png}
\begin{eulercomment}
r disini kita gunakan untuk mengatur rasio.


5. Gambarkan plot kontur untuk fungsi

\end{eulercomment}
\begin{eulerformula}
\[
z = \frac{1}{2}(x^2+y^2)
\]
\end{eulerformula}
\begin{eulercomment}
\end{eulercomment}
\begin{eulerprompt}
>plot3d("1/2*(x^2+y^2)",r=1,n=100,level="thin",>contour,>spectral,fscale=1,scale=1.1,>user):
\end{eulerprompt}
\eulerimg{22}{images/22305141029_EMT4Plot3D_Theresia Selvina-101.png}
\begin{eulercomment}
6. Gambarkanlah fungsi berikut

\end{eulercomment}
\begin{eulerformula}
\[
x^3+2x-7
\]
\end{eulerformula}
\begin{eulerprompt}
>function f(x) := x^3+2x-7; ...
>x=-1:0.01:1; t=linspace(0,2pi,50)'; ...
>Z=x; X=cos(t)*f(x); Y=sin(t)*f(x); ...
>pov3d(X,Y,Z,angle=40°,look=povlook(blue,0.1),height=10°,axis=0,zoom=5,light=[20,5,25]);
\end{eulerprompt}
\eulerimg{27}{images/22305141029_EMT4Plot3D_Theresia Selvina-103.png}
\begin{eulercomment}
kita memplot dengan koordinat. Kita memerlukan tiga matriks untuk
mendefinisikan objek tersebut. Kita memutar sebuah fungsi sekitar
sumbu z kemudian digambar menggunakan Povray.


7. Gambarlah fungsi\\
\end{eulercomment}
\begin{eulerformula}
\[
z=-x^2-y^2
\]
\end{eulerformula}
\begin{eulercomment}
dengan garis singgungnya
\end{eulercomment}
\begin{eulerprompt}
>z=-1:0.02:1.1; y=x'; z=-x^2-y^2;
>function myplot (x,y,z) ...
\end{eulerprompt}
\begin{eulerudf}
     zoom(2);
     wi=fullwindow()
     plotcontourplane(x,y,z,level="auto",<scale);
     plot3d(x,y,z,>hue,<scale,>add,color=white,level="thin");
     window(wi);
     reset();
  endfunction
\end{eulerudf}
\begin{eulerprompt}
>myplot(x,y,z):
\end{eulerprompt}
\begin{euleroutput}
  [0,  0,  1023,  1023]
\end{euleroutput}
\eulerimg{34}{images/22305141029_EMT4Plot3D_Theresia Selvina-105.png}
\begin{eulercomment}
8. Sketsakan

\end{eulercomment}
\begin{eulerformula}
\[
f(x,y)=-xy
\]
\end{eulerformula}
\begin{eulerprompt}
>function f(x,y):= -x*y
>pov3d("f");
\end{eulerprompt}
\eulerimg{27}{images/22305141029_EMT4Plot3D_Theresia Selvina-107.png}
\begin{eulercomment}
9. Gambarkan dua persamaan beikut dalam satu ruang\\
\end{eulercomment}
\begin{eulerformula}
\[
x^2+y^2=4
\]
\end{eulerformula}
\begin{eulerformula}
\[
\text{dan}
\]
\end{eulerformula}
\begin{eulerformula}
\[
y^2+z^2=4
\]
\end{eulerformula}
\begin{eulerprompt}
>plot3d("x^2+y^2-4",r=5,implicit=3);
>plot3d("y^2+z^2-4",r=5,implicit=3,>add):
\end{eulerprompt}
\eulerimg{34}{images/22305141029_EMT4Plot3D_Theresia Selvina-111.png}
\begin{eulercomment}
10. Untuk masing-masing fungsi sketsakan grafiknya

\end{eulercomment}
\begin{eulerformula}
\[
\text{a. } sin\sqrt{2x^2+y^2}
\]
\end{eulerformula}
\begin{eulerformula}
\[
\text{b. } 2x-y^2\exp{-x^2-y^2}
\]
\end{eulerformula}
\begin{eulerformula}
\[
\text{c. } xy\exp{-x^2-y^2}
\]
\end{eulerformula}
\begin{eulercomment}
Penyelesaian :

\end{eulercomment}
\begin{eulerformula}
\[
\text{a. } sin\sqrt{2x^2+y^2}
\]
\end{eulerformula}
\begin{eulerprompt}
>plot3d("sin(sqrt(2*x^2+y^2))",r=2):
\end{eulerprompt}
\eulerimg{34}{images/22305141029_EMT4Plot3D_Theresia Selvina-116.png}
\begin{eulercomment}
\end{eulercomment}
\begin{eulerformula}
\[
\text{b. } 2x-y^2\exp{-x^2-y^2}
\]
\end{eulerformula}
\begin{eulerprompt}
>plot3d("(2*x-y^2)*exp(-x^2-y^2)",r=2):
\end{eulerprompt}
\eulerimg{34}{images/22305141029_EMT4Plot3D_Theresia Selvina-118.png}
\begin{eulercomment}
\end{eulercomment}
\begin{eulerformula}
\[
\text{c. } xy\exp{-x^2-y^2}
\]
\end{eulerformula}
\begin{eulerprompt}
>plot3d("x*y*exp(-x^2-y^2)",r=2):
\end{eulerprompt}
\eulerimg{34}{images/22305141029_EMT4Plot3D_Theresia Selvina-120.png}
\begin{eulerudf}
  
  
  
  
  
  
  
\end{eulerudf}
\end{eulernotebook}
\end{document}
