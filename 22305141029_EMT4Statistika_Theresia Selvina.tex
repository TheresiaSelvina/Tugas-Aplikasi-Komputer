\documentclass[a4paper,10pt]{article}
\usepackage{eumat}

\begin{document}
\begin{eulernotebook}
\eulerheading{Tugas Aplikasi Komputer}
\begin{eulercomment}
Nama  : Theresia Selvina Vanny M.\\
NIM   : 22305141029\\
Kelas : Matematika B


\begin{eulercomment}
\eulerheading{EMT untuk Statistika}
\begin{eulercomment}
Dalam catatan ini, kami mendemonstrasikan plot statistik utama, uji,
dan distribusi utama dalam Euler. 

Mari kita mulai dengan beberapa statistik deskriptif. Ini bukan
pengantar statistik. Jadi Anda mungkin memerlukan beberapa pengetahuan
dasar untuk memahami detailnya. 

Asumsikan pengukuran berikut. Kami ingin menghitung nilai rata-rata
dan deviasi standar yang diukur.
\end{eulercomment}
\begin{eulerprompt}
>M=[1000,1004,998,997,1002,1001,998,1004,998,997]; ...
>median(M), mean(M), dev(M),
\end{eulerprompt}
\begin{euleroutput}
  999
  999.9
  2.72641400622
\end{euleroutput}
\begin{eulercomment}
Kita dapat membuat plot box-and-whisker untuk data. Dalam kasus kami,
tidak ada outlier.
\end{eulercomment}
\begin{eulerprompt}
>aspect(1.75); boxplot(M):
\end{eulerprompt}
\eulerimg{15}{images/22305141029_EMT4Statistika_Theresia Selvina-001.png}
\begin{eulercomment}
Kita menghitung probabilitas bahwa sebuah nilai lebih besar dari 1005,
dengan asumsi nilai-nilai yang diukur dan distribusi normal. 

Semua fungsi untuk distribusi di Euler diakhiri dengan ...dis dan
menghitung distribusi probabilitas kumulatif (CPF).

\end{eulercomment}
\begin{eulerformula}
\[
\text{normaldis(x,m,d)}=\int_{-\infty}^x \frac{1}{d\sqrt{2\pi}}e^{-\frac{1}{2}(\frac{t-m}{d})^2}\ dt.
\]
\end{eulerformula}
\begin{eulercomment}
Kami mencetak hasilnya dalam \% dengan akurasi 2 digit menggunakan
fungsi cetak.
\end{eulercomment}
\begin{eulerprompt}
>print((1-normaldis(1005,mean(M),dev(M)))*100,2,unit=" %")
\end{eulerprompt}
\begin{euleroutput}
        3.07 %
\end{euleroutput}
\begin{eulercomment}
Untuk contoh berikutnya, kami mengasumsikan jumlah pria dalam rentang
ukuran yang diberikan.
\end{eulercomment}
\begin{eulerprompt}
>r=155.5:4:187.5; v=[22,71,136,169,139,71,32,8];
\end{eulerprompt}
\begin{eulercomment}
Berikut adalah plot distribusi.
\end{eulercomment}
\begin{eulerprompt}
>plot2d(r,v,a=150,b=200,c=0,d=190,bar=1,style="\(\backslash\)/"):
\end{eulerprompt}
\eulerimg{15}{images/22305141029_EMT4Statistika_Theresia Selvina-003.png}
\begin{eulercomment}
Kami dapat memasukkan data mentah seperti ini ke dalam tabel. 

Tabel adalah metode untuk menyimpan data statistik. Tabel kami harus
berisi tiga kolom: Awal rentang, akhir rentang, jumlah pria dalam
rentang. 

Tabel dapat dicetak dengan header. Kami menggunakan vektor string
untuk mengatur header.
\end{eulercomment}
\begin{eulerprompt}
>T:=r[1:8]' | r[2:9]' | v'; writetable(T,labc=["BB","BA","Frek"])
\end{eulerprompt}
\begin{euleroutput}
          BB        BA      Frek
       155.5     159.5        22
       159.5     163.5        71
       163.5     167.5       136
       167.5     171.5       169
       171.5     175.5       139
       175.5     179.5        71
       179.5     183.5        32
       183.5     187.5         8
\end{euleroutput}
\begin{eulercomment}
Jika kita memerlukan nilai rata-rata dan statistik lainnya dari
ukuran-ukuran, kita perlu menghitung titik tengah dari rentang-rentang
tersebut. Kami dapat menggunakan dua kolom pertama dari tabel kami
untuk ini.

Sumbol "\textbar{}" digunakan untuk memisahkan kolom, dan fungsi "writetable"
digunakan untuk menulis tabel, dengan opsi "labc" adalah untuk
menentukan header kolom.
\end{eulercomment}
\begin{eulerprompt}
>(T[,1]+T[,2])/2 // the midpoint of each interval
\end{eulerprompt}
\begin{euleroutput}
          157.5 
          161.5 
          165.5 
          169.5 
          173.5 
          177.5 
          181.5 
          185.5 
\end{euleroutput}
\begin{eulercomment}
Tapi lebih mudah, untuk melipatgandakan rentang-rentang dengan vektor
[1/2,1/2].
\end{eulercomment}
\begin{eulerprompt}
>M=fold(r,[0.5,0.5])
\end{eulerprompt}
\begin{euleroutput}
  [157.5,  161.5,  165.5,  169.5,  173.5,  177.5,  181.5,  185.5]
\end{euleroutput}
\begin{eulercomment}
Sekarang kita dapat menghitung nilai rata-rata dan deviasi dari sampel
dengan frekuensi yang diberikan.
\end{eulercomment}
\begin{eulerprompt}
>\{m,d\}=meandev(M,v); m, d,
\end{eulerprompt}
\begin{euleroutput}
  169.901234568
  5.98912964449
\end{euleroutput}
\begin{eulercomment}
Mari kita tambahkan distribusi normal dari nilai-nilai di atas ke plot
batang di atas. Formula untuk distribusi normal dengan rata-rata m dan
deviasi standar d adalah:

\end{eulercomment}
\begin{eulerformula}
\[
y=\frac{1}{d\sqrt{2\pi}}e^{\frac{-(x-m)^2}{2d^2}}.
\]
\end{eulerformula}
\begin{eulercomment}
Karena nilainya antara 0 dan 1, untuk menggambarnya pada plot batang,
itu harus dikalikan dengan 4 kali jumlah total data.
\end{eulercomment}
\begin{eulerprompt}
>plot2d("qnormal(x,m,d)*sum(v)*4", ...
>  xmin=min(r),xmax=max(r),thickness=3,add=1):
\end{eulerprompt}
\eulerimg{15}{images/22305141029_EMT4Statistika_Theresia Selvina-005.png}
\eulerheading{Tabel}
\begin{eulercomment}
Di direktori catatan ini, Anda akan menemukan file dengan tabel. Data
tersebut mewakili hasil dari survei. Berikut adalah empat baris
pertama dari file tersebut. Data berasal dari buku online Jerman
"Einführung in die Statistik mit R" oleh A. Handl.
\end{eulercomment}
\begin{eulerprompt}
>printfile("table.dat",4);
\end{eulerprompt}
\begin{euleroutput}
  Person Sex Age Titanic Evaluation Tip Problem
  1 m 30 n . 1.80 n
  2 f 23 y g 1.80 n
  3 f 26 y g 1.80 y
\end{euleroutput}
\begin{eulercomment}
Tabel ini berisi 7 kolom angka atau token (string). Kami ingin membaca
tabel dari file tersebut. Pertama, kami menggunakan terjemahan sendiri
untuk token-token tersebut. 

Untuk ini, kami mendefinisikan set token. Fungsi strtokens()
mendapatkan vektor string token dari string yang diberikan.
\end{eulercomment}
\begin{eulerprompt}
>mf:=["m","f"]; yn:=["y","n"]; ev:=strtokens("g vg m b vb");
\end{eulerprompt}
\begin{eulercomment}
Sekarang kita membaca tabel dengan terjemahan ini. 

Argumen tok2, tok4, dll adalah terjemahan kolom tabel. Argumen-argumen
ini tidak ada dalam daftar parameter readtable(), jadi Anda perlu
memberikannya dengan ":=".
\end{eulercomment}
\begin{eulerprompt}
>\{MT,hd\}=readtable("table.dat",tok2:=mf,tok4:=yn,tok5:=ev,tok7:=yn);
>load over statistics;
\end{eulerprompt}
\begin{eulercomment}
Untuk mencetaknya, kita perlu menentukan set token yang sama. Kita
hanya mencetak empat baris pertama.
\end{eulercomment}
\begin{eulerprompt}
>writetable(MT[1:10],labc=hd,wc=5,tok2:=mf,tok4:=yn,tok5:=ev,tok7:=yn);
\end{eulerprompt}
\begin{euleroutput}
   Person  Sex  Age Titanic Evaluation  Tip Problem
        1    m   30       n          .  1.8       n
        2    f   23       y          g  1.8       n
        3    f   26       y          g  1.8       y
        4    m   33       n          .  2.8       n
        5    m   37       n          .  1.8       n
        6    m   28       y          g  2.8       y
        7    f   31       y         vg  2.8       n
        8    m   23       n          .  0.8       n
        9    f   24       y         vg  1.8       y
       10    m   26       n          .  1.8       n
\end{euleroutput}
\begin{eulercomment}
Titik "." mewakili nilai yang tidak tersedia.

Jika kita tidak ingin menentukan token untuk terjemahan sebelumnya,
kita hanya perlu menentukan kolom-kolom yang berisi token dan bukan
angka.
\end{eulercomment}
\begin{eulerprompt}
>ctok=[2,4,5,7]; \{MT,hd,tok\}=readtable("table.dat",ctok=ctok);
\end{eulerprompt}
\begin{eulercomment}
Fungsi readtable() sekarang mengembalikan set token.
\end{eulercomment}
\begin{eulerprompt}
>tok
\end{eulerprompt}
\begin{euleroutput}
  m
  n
  f
  y
  g
  vg
\end{euleroutput}
\begin{eulercomment}
Tabel ini berisi entri dari file dengan token yang diterjemahkan
menjadi angka. 

String khusus NA="." diinterpretasikan sebagai "Not Available", dan
mendapatkan NAN (not a number) dalam tabel. Terjemahan ini dapat
diubah dengan parameter NA dan NAval.
\end{eulercomment}
\begin{eulerprompt}
>MT[1]
\end{eulerprompt}
\begin{euleroutput}
  [1,  1,  30,  2,  NAN,  1.8,  2]
\end{euleroutput}
\begin{eulercomment}
Berikut ini konten tabel dengan angka yang belum diterjemahkan.
\end{eulercomment}
\begin{eulerprompt}
>writetable(MT,wc=5)
\end{eulerprompt}
\begin{euleroutput}
      1    1   30    2    .  1.8    2
      2    3   23    4    5  1.8    2
      3    3   26    4    5  1.8    4
      4    1   33    2    .  2.8    2
      5    1   37    2    .  1.8    2
      6    1   28    4    5  2.8    4
      7    3   31    4    6  2.8    2
      8    1   23    2    .  0.8    2
      9    3   24    4    6  1.8    4
     10    1   26    2    .  1.8    2
     11    3   23    4    6  1.8    4
     12    1   32    4    5  1.8    2
     13    1   29    4    6  1.8    4
     14    3   25    4    5  1.8    4
     15    3   31    4    5  0.8    2
     16    1   26    4    5  2.8    2
     17    1   37    2    .  3.8    2
     18    1   38    4    5    .    2
     19    3   29    2    .  3.8    2
     20    3   28    4    6  1.8    2
     21    3   28    4    1  2.8    4
     22    3   28    4    6  1.8    4
     23    3   38    4    5  2.8    2
     24    3   27    4    1  1.8    4
     25    1   27    2    .  2.8    4
\end{euleroutput}
\begin{eulercomment}
Untuk kenyamanan, Anda dapat menempatkan keluaran dari readtable() ke
dalam daftar.
\end{eulercomment}
\begin{eulerprompt}
>Table=\{\{readtable("table.dat",ctok=ctok)\}\};
\end{eulerprompt}
\begin{eulercomment}
Dengan kolom token yang sama dan token yang dibaca dari file, kami
dapat mencetak tabel. Kami dapat menentukan ctok, tok, dll. atau
menggunakan daftar Tabel.
\end{eulercomment}
\begin{eulerprompt}
>writetable(Table,ctok=ctok,wc=5);
\end{eulerprompt}
\begin{euleroutput}
   Person  Sex  Age Titanic Evaluation  Tip Problem
        1    m   30       n          .  1.8       n
        2    f   23       y          g  1.8       n
        3    f   26       y          g  1.8       y
        4    m   33       n          .  2.8       n
        5    m   37       n          .  1.8       n
        6    m   28       y          g  2.8       y
        7    f   31       y         vg  2.8       n
        8    m   23       n          .  0.8       n
        9    f   24       y         vg  1.8       y
       10    m   26       n          .  1.8       n
       11    f   23       y         vg  1.8       y
       12    m   32       y          g  1.8       n
       13    m   29       y         vg  1.8       y
       14    f   25       y          g  1.8       y
       15    f   31       y          g  0.8       n
       16    m   26       y          g  2.8       n
       17    m   37       n          .  3.8       n
       18    m   38       y          g    .       n
       19    f   29       n          .  3.8       n
       20    f   28       y         vg  1.8       n
       21    f   28       y          m  2.8       y
       22    f   28       y         vg  1.8       y
       23    f   38       y          g  2.8       n
       24    f   27       y          m  1.8       y
       25    m   27       n          .  2.8       y
\end{euleroutput}
\begin{eulercomment}
Fungsi tablecol() mengembalikan nilai kolom tabel, melompati baris
dengan nilai NAN ("." dalam file), dan indeks kolom yang berisi
nilai-nilai ini.
\end{eulercomment}
\begin{eulerprompt}
>\{c,i\}=tablecol(MT,[5,6]);
\end{eulerprompt}
\begin{eulercomment}
Kami dapat menggunakan ini untuk mengekstrak kolom dari tabel untuk
tabel baru.
\end{eulercomment}
\begin{eulerprompt}
>j=[1,5,6]; writetable(MT[i,j],labc=hd[j],ctok=[2],tok=tok)
\end{eulerprompt}
\begin{euleroutput}
      Person Evaluation       Tip
           2          g       1.8
           3          g       1.8
           6          g       2.8
           7         vg       2.8
           9         vg       1.8
          11         vg       1.8
          12          g       1.8
          13         vg       1.8
          14          g       1.8
          15          g       0.8
          16          g       2.8
          20         vg       1.8
          21          m       2.8
          22         vg       1.8
          23          g       2.8
          24          m       1.8
\end{euleroutput}
\begin{eulercomment}
Tentu saja, kita perlu mengekstrak tabel itu sendiri dari daftar Tabel
dalam kasus ini.
\end{eulercomment}
\begin{eulerprompt}
>MT=Table[1];
\end{eulerprompt}
\begin{eulercomment}
Tentu saja, kita juga dapat menggunakannya untuk menentukan nilai
rata-rata dari kolom atau nilai statistik lainnya.
\end{eulercomment}
\begin{eulerprompt}
>mean(tablecol(MT,6))
\end{eulerprompt}
\begin{euleroutput}
  2.175
\end{euleroutput}
\begin{eulercomment}
Fungsi getstatistics() mengembalikan elemen dalam vektor, dan
jumlahnya. Kami mengaplikasikannya pada nilai "m" dan "f" dalam kolom
kedua tabel kami.
\end{eulercomment}
\begin{eulerprompt}
>\{xu,count\}=getstatistics(tablecol(MT,2)); xu, count,
\end{eulerprompt}
\begin{euleroutput}
  [1,  3]
  [12,  13]
\end{euleroutput}
\begin{eulercomment}
Kami dapat mencetak hasilnya dalam tabel baru.
\end{eulercomment}
\begin{eulerprompt}
>writetable(count',labr=tok[xu])
\end{eulerprompt}
\begin{euleroutput}
           m        12
           f        13
\end{euleroutput}
\begin{eulercomment}
Fungsi selecttable() mengembalikan tabel baru dengan nilai dalam satu
kolom yang dipilih dari vektor indeks. Pertama kita mencari indeks dua
dari nilai kami dalam tabel token.
\end{eulercomment}
\begin{eulerprompt}
>v:=indexof(tok,["g","vg"])
\end{eulerprompt}
\begin{euleroutput}
  [5,  6]
\end{euleroutput}
\begin{eulercomment}
Sekarang kita bisa memilih baris dari tabel yang memiliki salah satu
dari nilai dalam v dalam baris ke-5 mereka.
\end{eulercomment}
\begin{eulerprompt}
>MT1:=MT[selectrows(MT,5,v)]; i:=sortedrows(MT1,5);
\end{eulerprompt}
\begin{eulercomment}
Sekarang kita bisa mencetak tabel, dengan nilai yang diekstrak dan
diurutkan dalam kolom ke-5.
\end{eulercomment}
\begin{eulerprompt}
>writetable(MT1[i],labc=hd,ctok=ctok,tok=tok,wc=7);
\end{eulerprompt}
\begin{euleroutput}
   Person    Sex    Age Titanic Evaluation    Tip Problem
        2      f     23       y          g    1.8       n
        3      f     26       y          g    1.8       y
        6      m     28       y          g    2.8       y
       18      m     38       y          g      .       n
       16      m     26       y          g    2.8       n
       15      f     31       y          g    0.8       n
       12      m     32       y          g    1.8       n
       23      f     38       y          g    2.8       n
       14      f     25       y          g    1.8       y
        9      f     24       y         vg    1.8       y
        7      f     31       y         vg    2.8       n
       20      f     28       y         vg    1.8       n
       22      f     28       y         vg    1.8       y
       13      m     29       y         vg    1.8       y
       11      f     23       y         vg    1.8       y
\end{euleroutput}
\begin{eulercomment}
Untuk statistik berikutnya, kita ingin mengaitkan dua kolom dalam
tabel. Jadi kita ekstrak kolom 2 dan 4, dan mengurutkan tabel.
\end{eulercomment}
\begin{eulerprompt}
>i=sortedrows(MT,[2,4]);  ...
>  writetable(tablecol(MT[i],[2,4])',ctok=[1,2],tok=tok)
\end{eulerprompt}
\begin{euleroutput}
           m         n
           m         n
           m         n
           m         n
           m         n
           m         n
           m         n
           m         y
           m         y
           m         y
           m         y
           m         y
           f         n
           f         y
           f         y
           f         y
           f         y
           f         y
           f         y
           f         y
           f         y
           f         y
           f         y
           f         y
           f         y
\end{euleroutput}
\begin{eulercomment}
Dengan getstatistics(), kita juga dapat mengaitkan jumlah dalam dua
kolom tabel satu sama lain.
\end{eulercomment}
\begin{eulerprompt}
>MT24=tablecol(MT,[2,4]); ...
>\{xu1,xu2,count\}=getstatistics(MT24[1],MT24[2]); ...
>writetable(count,labr=tok[xu1],labc=tok[xu2])
\end{eulerprompt}
\begin{euleroutput}
                     n         y
           m         7         5
           f         1        12
\end{euleroutput}
\begin{eulercomment}
Tabel dapat ditulis ke file.
\end{eulercomment}
\begin{eulerprompt}
>filename="test.dat"; ...
>writetable(count,labr=tok[xu1],labc=tok[xu2],file=filename);
\end{eulerprompt}
\begin{eulercomment}
Kemudian kita bisa membaca tabel dari file.
\end{eulercomment}
\begin{eulerprompt}
>\{MT2,hd,tok2,hdr\}=readtable(filename,>clabs,>rlabs); ...
>writetable(MT2,labr=hdr,labc=hd)
\end{eulerprompt}
\begin{euleroutput}
                     n         y
           m         7         5
           f         1        12
\end{euleroutput}
\begin{eulercomment}
Dan menghapus file.
\end{eulercomment}
\begin{eulerprompt}
>fileremove(filename);
\end{eulerprompt}
\eulerheading{Distribusi}
\begin{eulercomment}
Dengan plot2d, ada metode yang sangat mudah untuk menggambar
distribusi data eksperimental.
\end{eulercomment}
\begin{eulerprompt}
>p=normal(1,1000); //1000 random normal-distributed sample p
>plot2d(p,distribution=20,style="\(\backslash\)/"); // plot the random sample p
>plot2d("qnormal(x,0,1)",add=1): // add the standard normal distribution plot
\end{eulerprompt}
\eulerimg{15}{images/22305141029_EMT4Statistika_Theresia Selvina-006.png}
\begin{eulercomment}
Harap perhatikan perbedaan antara plot batang (sampel) dan kurva
normal (distribusi sebenarnya). Masukkan kembali tiga perintah ini
untuk melihat hasil sampel lain. 
\end{eulercomment}
\begin{eulercomment}
Berikut perbandingan dari 10 simulasi dari 1000 nilai yang
didistribusikan secara normal menggunakan apa yang disebut plot kotak.
Plot ini menunjukkan median, kuartil 25\% dan 75\%, nilai minimal dan
maksimal, dan outlier.
\end{eulercomment}
\begin{eulerprompt}
>p=normal(10,1000); boxplot(p):
\end{eulerprompt}
\eulerimg{15}{images/22305141029_EMT4Statistika_Theresia Selvina-007.png}
\begin{eulercomment}
Untuk menghasilkan angka acak, Euler memiliki intrandom. Mari kita
simulasi lemparan dadu dan gambar distribusinya. 

Kita menggunakan fungsi getmultiplicities(v,x), yang menghitung
seberapa sering elemen-elemen v muncul dalam x. Kemudian kami
menggambar hasilnya menggunakan columnsplot().
\end{eulercomment}
\begin{eulerprompt}
>k=intrandom(1,6000,6);  ...
>columnsplot(getmultiplicities(1:6,k));  ...
>ygrid(1000,color=red):
\end{eulerprompt}
\eulerimg{15}{images/22305141029_EMT4Statistika_Theresia Selvina-008.png}
\begin{eulercomment}
Sementara intrandom(n,m,k) mengembalikan bilangan bulat yang
didistribusikan secara merata dari 1 hingga k, mungkin ada kemungkinan
menggunakan distribusi bilangan bulat lainnya dengan randpint(). 

Pada contoh berikut, probabilitas untuk 1,2,3 adalah 0,4,0,1,0,5
masing-masing.
\end{eulercomment}
\begin{eulerprompt}
>randpint(1,1000,[0.4,0.1,0.5]); getmultiplicities(1:3,%)
\end{eulerprompt}
\begin{euleroutput}
  [378,  102,  520]
\end{euleroutput}
\begin{eulercomment}
Euler dapat menghasilkan nilai acak dari lebih banyak distribusi.
Lihatlah di referensi. 

Misalnya, kita mencoba distribusi eksponensial. Variabel acak kontinu
X dikatakan memiliki distribusi eksponensial, jika PDF-nya diberikan
oleh

\end{eulercomment}
\begin{eulerformula}
\[
f_X(x)=\lambda e^{-\lambda x},\quad x>0,\quad \lambda>0,
\]
\end{eulerformula}
\begin{eulercomment}
dengan parameter

\end{eulercomment}
\begin{eulerformula}
\[
\lambda=\frac{1}{\mu},\quad \mu \text{ is the mean, and denoted by } X \sim \text{Exponential}(\lambda).
\]
\end{eulerformula}
\begin{eulerprompt}
>plot2d(randexponential(1,1000,2),>distribution):
\end{eulerprompt}
\eulerimg{15}{images/22305141029_EMT4Statistika_Theresia Selvina-011.png}
\begin{eulercomment}
Untuk banyak distribusi, Euler dapat menghitung fungsi distribusi dan
inversnya.
\end{eulercomment}
\begin{eulerprompt}
>plot2d("normaldis",-4,4): 
\end{eulerprompt}
\eulerimg{15}{images/22305141029_EMT4Statistika_Theresia Selvina-012.png}
\begin{eulercomment}
Berikut adalah salah satu cara untuk menggambar kuantil.
\end{eulercomment}
\begin{eulerprompt}
>plot2d("qnormal(x,1,1.5)",-4,6);  ...
>plot2d("qnormal(x,1,1.5)",a=2,b=5,>add,>filled):
\end{eulerprompt}
\eulerimg{15}{images/22305141029_EMT4Statistika_Theresia Selvina-013.png}
\begin{eulerformula}
\[
\text{normaldis(x,m,d)}=\int_{-\infty}^x \frac{1}{d\sqrt{2\pi}}e^{-\frac{1}{2}(\frac{t-m}{d})^2}\ dt.
\]
\end{eulerformula}
\begin{eulercomment}
Probabilitas berada di area hijau adalah sebagai berikut.
\end{eulercomment}
\begin{eulerprompt}
>normaldis(5,1,1.5)-normaldis(2,1,1.5)
\end{eulerprompt}
\begin{euleroutput}
  0.248662156979
\end{euleroutput}
\begin{eulercomment}
Ini dapat dihitung secara numerik dengan integral berikut.

\end{eulercomment}
\begin{eulerformula}
\[
\int_2^5 \frac{1}{1.5\sqrt{2\pi}}e^{-\frac{1}{2}(\frac{x-1}{1.5})^2}\ dx.
\]
\end{eulerformula}
\begin{eulerprompt}
>gauss("qnormal(x,1,1.5)",2,5)
\end{eulerprompt}
\begin{euleroutput}
  0.248662156979
\end{euleroutput}
\begin{eulercomment}
Mari kita bandingkan distribusi binomial dengan distribusi normal
dengan rata-rata dan deviasi yang sama. Fungsi invbindis()
menyelesaikan interpolasi linier antara nilai-nilai bulat.
\end{eulercomment}
\begin{eulerprompt}
>invbindis(0.95,1000,0.5), invnormaldis(0.95,500,0.5*sqrt(1000))
\end{eulerprompt}
\begin{euleroutput}
  525.516721219
  526.007419394
\end{euleroutput}
\begin{eulercomment}
Fungsi qdis() adalah kepadatan distribusi chi-square. Seperti biasa,
Euler memetakan vektor ke fungsi ini. Dengan demikian, kita
mendapatkan plot semua distribusi chi-square dengan derajat 5 hingga
30 dengan mudah dengan cara berikut.
\end{eulercomment}
\begin{eulerprompt}
>plot2d("qchidis(x,(5:5:50)')",0,50):
\end{eulerprompt}
\eulerimg{15}{images/22305141029_EMT4Statistika_Theresia Selvina-016.png}
\begin{eulercomment}
Euler memiliki fungsi yang akurat untuk mengevaluasi distribusi. Mari
kita periksa chidis() dengan integral. 

Nama-nama mencoba untuk konsisten. Misalnya,

- distribusi chi-square adalah chidis(),\\
- fungsi invers adalah invchidis(),\\
- kepadatan adalah qchidis().

Komplemen dari distribusi (ekor atas) adalah chicdis().
\end{eulercomment}
\begin{eulerprompt}
>chidis(1.5,2), integrate("qchidis(x,2)",0,1.5)
\end{eulerprompt}
\begin{euleroutput}
  0.527633447259
  0.527633447259
\end{euleroutput}
\eulerheading{Distribusi Diskrit}
\begin{eulercomment}
Untuk mendefinisikan distribusi diskrit Anda sendiri, Anda dapat
menggunakan metode berikut. 

Pertama, kita mengatur fungsi distribusi.
\end{eulercomment}
\begin{eulerprompt}
>wd = 0|((1:6)+[-0.01,0.01,0,0,0,0])/6
\end{eulerprompt}
\begin{euleroutput}
  [0,  0.165,  0.335,  0.5,  0.666667,  0.833333,  1]
\end{euleroutput}
\begin{eulercomment}
Artinya adalah bahwa dengan probabilitas wd[i+1]-wd[i], kita
menghasilkan nilai acak i. 

Ini hampir merupakan distribusi seragam. Mari kita definisikan
pembangkit angka acak untuk ini. Fungsi find(v,x) mencari nilai x
dalam vektor v. Ini juga berfungsi untuk vektor x.
\end{eulercomment}
\begin{eulerprompt}
>function wrongdice (n,m) := find(wd,random(n,m))
\end{eulerprompt}
\begin{eulercomment}
Kesalahan ini sangat halus sehingga kita hanya melihatnya dengan
banyak iterasi.
\end{eulercomment}
\begin{eulerprompt}
>columnsplot(getmultiplicities(1:6,wrongdice(1,1000000))):
\end{eulerprompt}
\eulerimg{15}{images/22305141029_EMT4Statistika_Theresia Selvina-017.png}
\begin{eulercomment}
Berikut adalah fungsi sederhana untuk memeriksa distribusi seragam
dari nilai 1...K dalam v. Kami menerima hasilnya jika untuk semua
frekuensi

\end{eulercomment}
\begin{eulerformula}
\[
\left|f_i-\frac{1}{K}\right| < \frac{\delta}{\sqrt{n}}.
\]
\end{eulerformula}
\begin{eulerprompt}
>function checkrandom (v, delta=1) ...
\end{eulerprompt}
\begin{eulerudf}
    K=max(v); n=cols(v);
    fr=getfrequencies(v,1:K);
    return max(fr/n-1/K)<delta/sqrt(n);
    endfunction
\end{eulerudf}
\begin{eulercomment}
Memang fungsi ini menolak distribusi seragam.
\end{eulercomment}
\begin{eulerprompt}
>checkrandom(wrongdice(1,1000000))
\end{eulerprompt}
\begin{euleroutput}
  0
\end{euleroutput}
\begin{eulercomment}
Dan ini menerima generator acak bawaan.
\end{eulercomment}
\begin{eulerprompt}
>checkrandom(intrandom(1,1000000,6))
\end{eulerprompt}
\begin{euleroutput}
  1
\end{euleroutput}
\begin{eulercomment}
Kita dapat menghitung distribusi binomial. Pertama ada binomialsum(),
yang mengembalikan probabilitas i atau lebih hasil dari n percobaan.
\end{eulercomment}
\begin{eulerprompt}
>bindis(410,1000,0.4)
\end{eulerprompt}
\begin{euleroutput}
  0.751401349654
\end{euleroutput}
\begin{eulercomment}
Fungsi invers Beta digunakan untuk menghitung interval kepercayaan
Clopper-Pearson untuk parameter p. Tingkat default adalah alpha.

Makna interval ini adalah jika p berada di luar interval, hasil yang
diamati sebesar 410 dari 1000 adalah langka.
\end{eulercomment}
\begin{eulerprompt}
>clopperpearson(410,1000)
\end{eulerprompt}
\begin{euleroutput}
  [0.37932,  0.441212]
\end{euleroutput}
\begin{eulercomment}
Perintah berikut adalah cara langsung untuk mendapatkan hasil di atas.
Tetapi untuk n besar, penjumlahan langsung tidak akurat dan lambat.
\end{eulercomment}
\begin{eulerprompt}
>p=0.4; i=0:410; n=1000; sum(bin(n,i)*p^i*(1-p)^(n-i))
\end{eulerprompt}
\begin{euleroutput}
  0.751401349655
\end{euleroutput}
\begin{eulercomment}
Sekadar informasi, invbinsum() menghitung invers dari binomialsum().
\end{eulercomment}
\begin{eulerprompt}
>invbindis(0.75,1000,0.4)
\end{eulerprompt}
\begin{euleroutput}
  409.932733047
\end{euleroutput}
\begin{eulercomment}
Dalam Bridge, kita mengasumsikan 5 kartu terbuka (dari 52) dalam dua
tangan (26 kartu). Mari kita hitung probabilitas distribusi yang lebih
buruk daripada 3:2 (misalnya, 0:5, 1:4, 4:1, atau 5:0).
\end{eulercomment}
\begin{eulerprompt}
>2*hypergeomsum(1,5,13,26)
\end{eulerprompt}
\begin{euleroutput}
  0.321739130435
\end{euleroutput}
\begin{eulercomment}
Ada juga simulasi distribusi multinomial.
\end{eulercomment}
\begin{eulerprompt}
>randmultinomial(10,1000,[0.4,0.1,0.5])
\end{eulerprompt}
\begin{euleroutput}
            381           100           519 
            376            91           533 
            417            80           503 
            440            94           466 
            406           112           482 
            408            94           498 
            395           107           498 
            399            96           505 
            428            87           485 
            400            99           501 
\end{euleroutput}
\eulerheading{Plot Data}
\begin{eulercomment}
Untuk memplot data, kita mencoba hasil pemilihan Jerman sejak tahun
1990, diukur dalam jumlah kursi.
\end{eulercomment}
\begin{eulerprompt}
>BW := [ ...
>1990,662,319,239,79,8,17; ...
>1994,672,294,252,47,49,30; ...
>1998,669,245,298,43,47,36; ...
>2002,603,248,251,47,55,2; ...
>2005,614,226,222,61,51,54; ...
>2009,622,239,146,93,68,76; ...
>2013,631,311,193,0,63,64];
\end{eulerprompt}
\begin{eulercomment}
Untuk partai-partai, kita menggunakan string nama-nama.
\end{eulercomment}
\begin{eulerprompt}
>P:=["CDU/CSU","SPD","FDP","Gr","Li"];
\end{eulerprompt}
\begin{eulercomment}
Mari kita mencetak persentasenya dengan baik.

Pertama, kita ekstrak kolom yang diperlukan. Kolom 3 hingga 7 adalah
kursi dari setiap partai, dan kolom 2 adalah jumlah total kursi. Kolom
adalah tahun pemilu.
\end{eulercomment}
\begin{eulerprompt}
>BT:=BW[,3:7]; BT:=BT/sum(BT); YT:=BW[,1]';
\end{eulerprompt}
\begin{eulercomment}
Kemudian kita mencetak statistik dalam bentuk tabel. Kami menggunakan
nama-nama sebagai header kolom, dan tahun-tahun sebagai header untuk
baris. Lebar default untuk kolom adalah wc=10, tetapi kita lebih suka
keluaran yang lebih padat. Kolom akan diperluas untuk label kolom,
jika diperlukan.
\end{eulercomment}
\begin{eulerprompt}
>writetable(BT*100,wc=6,dc=0,>fixed,labc=P,labr=YT)
\end{eulerprompt}
\begin{euleroutput}
         CDU/CSU   SPD   FDP    Gr    Li
    1990      48    36    12     1     3
    1994      44    38     7     7     4
    1998      37    45     6     7     5
    2002      41    42     8     9     0
    2005      37    36    10     8     9
    2009      38    23    15    11    12
    2013      49    31     0    10    10
\end{euleroutput}
\begin{eulercomment}
Perkalian matriks berikutnya mengekstraksi jumlah persentase dua
partai besar, menunjukkan bahwa partai-partai kecil telah mendapatkan
kursi di parlemen hingga tahun 2009.
\end{eulercomment}
\begin{eulerprompt}
>BT1:=(BT.[1;1;0;0;0])'*100
\end{eulerprompt}
\begin{euleroutput}
  [84.29,  81.25,  81.1659,  82.7529,  72.9642,  61.8971,  79.8732]
\end{euleroutput}
\begin{eulercomment}
Ada juga plot statistik sederhana. Kami menggunakannya untuk
menampilkan garis dan titik secara bersamaan. Alternatifnya adalah
memanggil plot2d dua kali dengan \textgreater{}add.
\end{eulercomment}
\begin{eulerprompt}
>statplot(YT,BT1,"b"):
\end{eulerprompt}
\eulerimg{15}{images/22305141029_EMT4Statistika_Theresia Selvina-019.png}
\begin{eulercomment}
Tentukan beberapa warna untuk setiap partai.
\end{eulercomment}
\begin{eulerprompt}
>CP:=[rgb(0.5,0.5,0.5),red,yellow,green,rgb(0.8,0,0)];
\end{eulerprompt}
\begin{eulercomment}
Sekarang kita bisa memplot hasil pemilihan tahun 2009 dan perubahan
dalam satu plot menggunakan figure. Kita bisa menambahkan vektor kolom
ke masing-masing plot.
\end{eulercomment}
\begin{eulerprompt}
>figure(2,1);  ...
>figure(1); columnsplot(BW[6,3:7],P,color=CP); ...
>figure(2); columnsplot(BW[6,3:7]-BW[5,3:7],P,color=CP);  ...
>figure(0):
\end{eulerprompt}
\eulerimg{15}{images/22305141029_EMT4Statistika_Theresia Selvina-020.png}
\begin{eulercomment}
Plot data menggabungkan baris data statistik dalam satu plot.
\end{eulercomment}
\begin{eulerprompt}
>J:=BW[,1]'; DP:=BW[,3:7]'; ...
>dataplot(YT,BT',color=CP);  ...
>labelbox(P,colors=CP,styles="[]",>points,w=0.2,x=0.3,y=0.4):
\end{eulerprompt}
\eulerimg{15}{images/22305141029_EMT4Statistika_Theresia Selvina-021.png}
\begin{eulercomment}
Plot kolom 3D menampilkan baris data statistik dalam bentuk kolom.
Kami memberikan label untuk baris dan kolom. Sudut adalah sudut
pandangan.
\end{eulercomment}
\begin{eulerprompt}
>columnsplot3d(BT,scols=P,srows=YT, ...
>  angle=30°,ccols=CP):
\end{eulerprompt}
\eulerimg{15}{images/22305141029_EMT4Statistika_Theresia Selvina-022.png}
\begin{eulercomment}
Representasi lainnya adalah plot mozaik. Perhatikan bahwa kolom plot
mewakili kolom matriks di sini. Karena panjang label CDU/CSU, kita
mengambil jendela yang lebih kecil dari biasanya.
\end{eulercomment}
\begin{eulerprompt}
>shrinkwindow(>smaller);  ...
>mosaicplot(BT',srows=YT,scols=P,color=CP,style="#"); ...
>shrinkwindow():
\end{eulerprompt}
\eulerimg{15}{images/22305141029_EMT4Statistika_Theresia Selvina-023.png}
\begin{eulercomment}
Kita juga bisa membuat pie chart. Karena hitam dan kuning membentuk
koalisi, kita mengurutkan ulang elemen-elemen.
\end{eulercomment}
\begin{eulerprompt}
>i=[1,3,5,4,2]; piechart(BW[6,3:7][i],color=CP[i],lab=P[i]):
\end{eulerprompt}
\eulerimg{15}{images/22305141029_EMT4Statistika_Theresia Selvina-024.png}
\begin{eulercomment}
Berikut adalah jenis plot lainnya.
\end{eulercomment}
\begin{eulerprompt}
>starplot(normal(1,10)+4,lab=1:10,>rays):
\end{eulerprompt}
\eulerimg{15}{images/22305141029_EMT4Statistika_Theresia Selvina-025.png}
\begin{eulercomment}
Beberapa plot dalam plot2d cocok untuk statistik. Berikut adalah plot
impuls dari data acak, didistribusikan secara seragam dalam [0,1].
\end{eulercomment}
\begin{eulerprompt}
>plot2d(makeimpulse(1:10,random(1,10)),>bar):
\end{eulerprompt}
\eulerimg{15}{images/22305141029_EMT4Statistika_Theresia Selvina-026.png}
\begin{eulercomment}
Tetapi untuk data yang didistribusikan secara eksponensial, kita
mungkin memerlukan plot logaritmik.
\end{eulercomment}
\begin{eulerprompt}
>logimpulseplot(1:10,-log(random(1,10))*10):
\end{eulerprompt}
\eulerimg{15}{images/22305141029_EMT4Statistika_Theresia Selvina-027.png}
\begin{eulercomment}
Fungsi columnsplot() lebih mudah digunakan, karena hanya memerlukan
vektor nilai. Selain itu, ia dapat mengatur labelnya menjadi apa pun
yang kita inginkan, seperti yang telah kami tunjukkan dalam tutorial
ini.

Berikut adalah aplikasi lain, di mana kita menghitung karakter dalam
sebuah kalimat dan memplot statistik.
\end{eulercomment}
\begin{eulerprompt}
>v=strtochar("the quick brown fox jumps over the lazy dog"); ...
>w=ascii("a"):ascii("z"); x=getmultiplicities(w,v); ...
>cw=[]; for k=w; cw=cw|char(k); end; ...
>columnsplot(x,lab=cw,width=0.05):
\end{eulerprompt}
\eulerimg{15}{images/22305141029_EMT4Statistika_Theresia Selvina-028.png}
\begin{eulercomment}
Juga memungkinkan untuk secara manual menetapkan sumbu.
\end{eulercomment}
\begin{eulerprompt}
>n=10; p=0.4; i=0:n; x=bin(n,i)*p^i*(1-p)^(n-i); ...
>columnsplot(x,lab=i,width=0.05,<frame,<grid); ...
>yaxis(0,0:0.1:1,style="->",>left); xaxis(0,style="."); ...
>label("p",0,0.25), label("i",11,0); ...
>textbox(["Binomial distribution","with p=0.4"]):
\end{eulerprompt}
\eulerimg{15}{images/22305141029_EMT4Statistika_Theresia Selvina-029.png}
\begin{eulercomment}
Berikut adalah cara untuk memplot frekuensi angka dalam vektor. 

Kami membuat vektor angka acak bulat 1 hingga 6.
\end{eulercomment}
\begin{eulerprompt}
>v:=intrandom(1,10,10)
\end{eulerprompt}
\begin{euleroutput}
  [8,  5,  8,  8,  6,  8,  8,  3,  5,  5]
\end{euleroutput}
\begin{eulercomment}
Kemudian ekstrak angka-angka unik dalam v.
\end{eulercomment}
\begin{eulerprompt}
>vu:=unique(v)
\end{eulerprompt}
\begin{euleroutput}
  [3,  5,  6,  8]
\end{euleroutput}
\begin{eulercomment}
Dan memplot frekuensi dalam plot kolom.
\end{eulercomment}
\begin{eulerprompt}
>columnsplot(getmultiplicities(vu,v),lab=vu,style="/"):
\end{eulerprompt}
\eulerimg{15}{images/22305141029_EMT4Statistika_Theresia Selvina-030.png}
\begin{eulercomment}
Kami ingin mendemonstrasikan fungsi distribusi empiris nilai.
\end{eulercomment}
\begin{eulerprompt}
>x=normal(1,20);
\end{eulerprompt}
\begin{eulercomment}
Fungsi empdist(x,vs) memerlukan array nilai yang sudah diurutkan. Jadi
kita harus mengurutkan x sebelum menggunakannya.
\end{eulercomment}
\begin{eulerprompt}
>xs=sort(x);
\end{eulerprompt}
\begin{eulercomment}
Kemudian kita memplot distribusi empiris dan beberapa bar densitas
dalam satu plot. Alih-alih plot kolom untuk distribusi, kita kali ini
menggunakan plot gergaji.
\end{eulercomment}
\begin{eulerprompt}
>figure(2,1); ...
>figure(1); plot2d("empdist",-4,4;xs); ...
>figure(2); plot2d(histo(x,v=-4:0.2:4,<bar));  ...
>figure(0):
\end{eulerprompt}
\eulerimg{15}{images/22305141029_EMT4Statistika_Theresia Selvina-031.png}
\begin{eulercomment}
Plot scatter mudah dilakukan dalam Euler dengan plot titik biasa.
Grafik berikut menunjukkan bahwa X dan X+Y jelas berkorelasi positif.
\end{eulercomment}
\begin{eulerprompt}
>x=normal(1,100); plot2d(x,x+rotright(x),>points,style=".."):
\end{eulerprompt}
\eulerimg{15}{images/22305141029_EMT4Statistika_Theresia Selvina-032.png}
\begin{eulercomment}
Seringkali, kita ingin membandingkan dua sampel dari distribusi yang
berbeda. Ini dapat dilakukan dengan plot kuantil-kuantil.

Untuk pengujian, kita mencoba distribusi student-t dan distribusi
eksponensial.
\end{eulercomment}
\begin{eulerprompt}
>x=randt(1,1000,5); y=randnormal(1,1000,mean(x),dev(x)); ...
>plot2d("x",r=6,style="--",yl="normal",xl="student-t",>vertical); ...
>plot2d(sort(x),sort(y),>points,color=red,style="x",>add):
\end{eulerprompt}
\eulerimg{15}{images/22305141029_EMT4Statistika_Theresia Selvina-033.png}
\begin{eulercomment}
Plot ini dengan jelas menunjukkan bahwa nilai yang didistribusikan
secara normal cenderung lebih kecil di ujung-ujung ekstrim.

Jika kita memiliki dua distribusi ukuran yang berbeda, kita dapat
memperluas yang lebih kecil atau mengecilkan yang lebih besar. Fungsi
berikut baik untuk keduanya. Ini mengambil nilai median dengan
persentase antara 0 dan 1.
\end{eulercomment}
\begin{eulerprompt}
>function medianexpand (x,n) := median(x,p=linspace(0,1,n-1));
\end{eulerprompt}
\begin{eulercomment}
Mari kita bandingkan dua distribusi yang sama.
\end{eulercomment}
\begin{eulerprompt}
>x=random(1000); y=random(400); ...
>plot2d("x",0,1,style="--"); ...
>plot2d(sort(medianexpand(x,400)),sort(y),>points,color=red,style="x",>add):
\end{eulerprompt}
\eulerimg{15}{images/22305141029_EMT4Statistika_Theresia Selvina-034.png}
\eulerheading{Regresi dan Korelasi}
\begin{eulercomment}
Regresi linear dapat dilakukan dengan fungsi polyfit() atau berbagai
fungsi fit. 

Untuk awal, kita menemukan garis regresi untuk data univariat dengan
polyfit(x, y, 1).
\end{eulercomment}
\begin{eulerprompt}
>x=1:10; y=[2,3,1,5,6,3,7,8,9,8]; writetable(x'|y',labc=["x","y"])
\end{eulerprompt}
\begin{euleroutput}
           x         y
           1         2
           2         3
           3         1
           4         5
           5         6
           6         3
           7         7
           8         8
           9         9
          10         8
\end{euleroutput}
\begin{eulercomment}
Kita ingin membandingkan fit non-bobot dan fit berbobot. Pertama,
koefisien dari fit linier.
\end{eulercomment}
\begin{eulerprompt}
>p=polyfit(x,y,1)
\end{eulerprompt}
\begin{euleroutput}
  [0.733333,  0.812121]
\end{euleroutput}
\begin{eulercomment}
Sekarang koefisien dengan bobot yang menekankan nilai terakhir.
\end{eulercomment}
\begin{eulerprompt}
>w &= "exp(-(x-10)^2/10)"; pw=polyfit(x,y,1,w=w(x))
\end{eulerprompt}
\begin{euleroutput}
  [4.71566,  0.38319]
\end{euleroutput}
\begin{eulercomment}
Kita menempatkan semuanya dalam satu plot untuk titik-titik dan garis
regresi, dan untuk bobot yang digunakan.
\end{eulercomment}
\begin{eulerprompt}
>figure(2,1);  ...
>figure(1); statplot(x,y,"b",xl="Regression"); ...
>  plot2d("evalpoly(x,p)",>add,color=blue,style="--"); ...
>  plot2d("evalpoly(x,pw)",5,10,>add,color=red,style="--"); ...
>figure(2); plot2d(w,1,10,>filled,style="/",fillcolor=red,xl=w); ...
>figure(0):
\end{eulerprompt}
\eulerimg{15}{images/22305141029_EMT4Statistika_Theresia Selvina-035.png}
\begin{eulercomment}
Untuk contoh lain, kita membaca hasil survei siswa, usia mereka, usia
orangtua mereka, dan jumlah saudara kandung dari file.

Tabel ini berisi "m" dan "f" dalam kolom kedua. Kami menggunakan
variabel tok2 untuk mengatur terjemahan yang tepat daripada membiarkan
readtable() mengumpulkan terjemahan.
\end{eulercomment}
\begin{eulerprompt}
>\{MS,hd\}:=readtable("table1.dat",tok2:=["m","f"]);  ...
>writetable(MS,labc=hd,tok2:=["m","f"]);
\end{eulerprompt}
\begin{euleroutput}
      Person       Sex       Age    Mother    Father  Siblings
           1         m        29        58        61         1
           2         f        26        53        54         2
           3         m        24        49        55         1
           4         f        25        56        63         3
           5         f        25        49        53         0
           6         f        23        55        55         2
           7         m        23        48        54         2
           8         m        27        56        58         1
           9         m        25        57        59         1
          10         m        24        50        54         1
          11         f        26        61        65         1
          12         m        24        50        52         1
          13         m        29        54        56         1
          14         m        28        48        51         2
          15         f        23        52        52         1
          16         m        24        45        57         1
          17         f        24        59        63         0
          18         f        23        52        55         1
          19         m        24        54        61         2
          20         f        23        54        55         1
\end{euleroutput}
\begin{eulercomment}
Bagaimana usia saling bergantung? Kesimpulan pertama datang dari
pairwise scatterplot.
\end{eulercomment}
\begin{eulerprompt}
>scatterplots(tablecol(MS,3:5),hd[3:5]):
\end{eulerprompt}
\eulerimg{15}{images/22305141029_EMT4Statistika_Theresia Selvina-036.png}
\begin{eulercomment}
Jelas bahwa usia ayah dan ibu saling bergantung. Mari kita tentukan
dan plot garis regresi.
\end{eulercomment}
\begin{eulerprompt}
>cs:=MS[,4:5]'; ps:=polyfit(cs[1],cs[2],1)
\end{eulerprompt}
\begin{euleroutput}
  [17.3789,  0.740964]
\end{euleroutput}
\begin{eulercomment}
Ini jelas adalah model yang salah. Garis regresi akan menjadi
s=17+0,74t, di mana t adalah usia ibu dan s adalah usia ayah.
Perbedaan usia mungkin sedikit bergantung pada usia, tetapi tidak
sebanyak itu.

Lebih tepatnya, kita menduga fungsi seperti s=a+t. Maka a adalah
rata-rata dari s-t. Itu adalah selisih usia rata-rata antara ayah dan
ibu.
\end{eulercomment}
\begin{eulerprompt}
>da:=mean(cs[2]-cs[1])
\end{eulerprompt}
\begin{euleroutput}
  3.65
\end{euleroutput}
\begin{eulercomment}
Mari kita plot ini dalam satu scatter plot.
\end{eulercomment}
\begin{eulerprompt}
>plot2d(cs[1],cs[2],>points);  ...
>plot2d("evalpoly(x,ps)",color=red,style=".",>add);  ...
>plot2d("x+da",color=blue,>add):
\end{eulerprompt}
\eulerimg{15}{images/22305141029_EMT4Statistika_Theresia Selvina-037.png}
\begin{eulercomment}
Berikut adalah plot kotak dari dua usia tersebut. Ini hanya
menunjukkan bahwa usia mereka berbeda.
\end{eulercomment}
\begin{eulerprompt}
>boxplot(cs,["mothers","fathers"]):
\end{eulerprompt}
\eulerimg{15}{images/22305141029_EMT4Statistika_Theresia Selvina-038.png}
\begin{eulercomment}
Menarik bahwa perbedaan median tidak sebesar perbedaan rata-rata.
\end{eulercomment}
\begin{eulerprompt}
>median(cs[2])-median(cs[1])
\end{eulerprompt}
\begin{euleroutput}
  1.5
\end{euleroutput}
\begin{eulercomment}
Koefisien korelasi menunjukkan korelasi positif.
\end{eulercomment}
\begin{eulerprompt}
>correl(cs[1],cs[2])
\end{eulerprompt}
\begin{euleroutput}
  0.7588307236
\end{euleroutput}
\begin{eulercomment}
Korelasi peringkat adalah ukuran urutan yang sama dalam kedua vektor.
Ini juga cukup positif.
\end{eulercomment}
\begin{eulerprompt}
>rankcorrel(cs[1],cs[2])
\end{eulerprompt}
\begin{euleroutput}
  0.758925292358
\end{euleroutput}
\eulerheading{Membuat Fungsi Baru}
\begin{eulercomment}
Tentu saja, bahasa EMT dapat digunakan untuk memprogram fungsi baru.
Misalnya, kita mendefinisikan fungsi skewness.

\end{eulercomment}
\begin{eulerformula}
\[
\text{sk}(x) = \dfrac{\sqrt{n} \sum_i (x_i-m)^3}{\left(\sum_i (x_i-m)^2\right)^{3/2}}
\]
\end{eulerformula}
\begin{eulercomment}
di mana m adalah rata-rata x.
\end{eulercomment}
\begin{eulerprompt}
>function skew (x:vector) ...
\end{eulerprompt}
\begin{eulerudf}
  m=mean(x);
  return sqrt(cols(x))*sum((x-m)^3)/(sum((x-m)^2))^(3/2);
  endfunction
\end{eulerudf}
\begin{eulercomment}
Seperti yang Anda lihat, kita dapat dengan mudah menggunakan bahasa
matriks untuk mendapatkan implementasi yang sangat singkat dan
efisien. Mari kita coba fungsi ini.
\end{eulercomment}
\begin{eulerprompt}
>data=normal(20); skew(normal(10))
\end{eulerprompt}
\begin{euleroutput}
  -0.198710316203
\end{euleroutput}
\begin{eulercomment}
Berikut adalah fungsi lain, yang disebut koefisien skewness Pearson.
\end{eulercomment}
\begin{eulerprompt}
>function skew1 (x) := 3*(mean(x)-median(x))/dev(x)
>skew1(data)
\end{eulerprompt}
\begin{euleroutput}
  -0.0801873249135
\end{euleroutput}
\eulerheading{Simulasi Monte Carlo}
\begin{eulercomment}
Euler dapat digunakan untuk mensimulasikan peristiwa acak. Kita telah
melihat contoh-contoh sederhana di atas. Berikut adalah contoh lain,
yang mensimulasikan 1000 kali lemparan tiga dadu, dan bertanya tentang
distribusi jumlahnya.
\end{eulercomment}
\begin{eulerprompt}
>ds:=sum(intrandom(1000,3,6))';  fs=getmultiplicities(3:18,ds)
\end{eulerprompt}
\begin{euleroutput}
  [5,  17,  35,  44,  75,  97,  114,  116,  143,  116,  104,  53,  40,
  22,  13,  6]
\end{euleroutput}
\begin{eulercomment}
Sekarang kita bisa memplotnya.
\end{eulercomment}
\begin{eulerprompt}
>columnsplot(fs,lab=3:18):
\end{eulerprompt}
\eulerimg{15}{images/22305141029_EMT4Statistika_Theresia Selvina-040.png}
\begin{eulercomment}
Untuk menentukan distribusi yang diharapkan tidak semudah itu. Kami
menggunakan rekurasi lanjutan untuk ini. 

Fungsi berikut menghitung jumlah cara di mana angka k dapat
direpresentasikan sebagai jumlah n angka dalam rentang 1 hingga m. Ini
bekerja secara rekursif dengan cara yang jelas.
\end{eulercomment}
\begin{eulerprompt}
>function map countways (k; n, m) ...
\end{eulerprompt}
\begin{eulerudf}
    if n==1 then return k>=1 && k<=m
    else
      sum=0; 
      loop 1 to m; sum=sum+countways(k-#,n-1,m); end;
      return sum;
    end;
  endfunction
\end{eulerudf}
\begin{eulercomment}
Berikut hasilnya untuk tiga lemparan dadu.
\end{eulercomment}
\begin{eulerprompt}
>countways(5:25,5,5)
\end{eulerprompt}
\begin{euleroutput}
  [1,  5,  15,  35,  70,  121,  185,  255,  320,  365,  381,  365,  320,
  255,  185,  121,  70,  35,  15,  5,  1]
\end{euleroutput}
\begin{eulerprompt}
>cw=countways(3:18,3,6)
\end{eulerprompt}
\begin{euleroutput}
  [1,  3,  6,  10,  15,  21,  25,  27,  27,  25,  21,  15,  10,  6,  3,
  1]
\end{euleroutput}
\begin{eulercomment}
Kami menambahkan nilai yang diharapkan ke plot.
\end{eulercomment}
\begin{eulerprompt}
>plot2d(cw/6^3*1000,>add); plot2d(cw/6^3*1000,>points,>add):
\end{eulerprompt}
\eulerimg{15}{images/22305141029_EMT4Statistika_Theresia Selvina-041.png}
\begin{eulercomment}
Untuk simulasi lain, deviasi dari nilai rata-rata n variabel acak yang
didistribusikan secara normal 0-1 adalah 1/sqrt(n).
\end{eulercomment}
\begin{eulerprompt}
>longformat; 1/sqrt(10)
\end{eulerprompt}
\begin{euleroutput}
  0.316227766017
\end{euleroutput}
\begin{eulercomment}
Mari kita periksa ini dengan simulasi. Kami menghasilkan 10000 kali 10
vektor acak.
\end{eulercomment}
\begin{eulerprompt}
>M=normal(10000,10); dev(mean(M)')
\end{eulerprompt}
\begin{euleroutput}
  0.319493614817
\end{euleroutput}
\begin{eulerprompt}
>plot2d(mean(M)',>distribution):
\end{eulerprompt}
\eulerimg{15}{images/22305141029_EMT4Statistika_Theresia Selvina-042.png}
\begin{eulercomment}
Median dari 10 angka acak yang didistribusikan secara normal 0-1
memiliki deviasi yang lebih besar.
\end{eulercomment}
\begin{eulerprompt}
>dev(median(M)')
\end{eulerprompt}
\begin{euleroutput}
  0.374460271535
\end{euleroutput}
\begin{eulercomment}
Karena kita dapat dengan mudah menghasilkan perjalanan acak, kita
dapat mensimulasikan proses Wiener. Kami mengambil 1000 langkah dari
1000 proses. Kemudian kita memplot deviasi standar dan rata-rata
langkah ke-n dari proses-proses ini bersama dengan nilai yang
diharapkan dalam merah.
\end{eulercomment}
\begin{eulerprompt}
>n=1000; m=1000; M=cumsum(normal(n,m)/sqrt(m)); ...
>t=(1:n)/n; figure(2,1); ...
>figure(1); plot2d(t,mean(M')'); plot2d(t,0,color=red,>add); ...
>figure(2); plot2d(t,dev(M')'); plot2d(t,sqrt(t),color=red,>add); ...
>figure(0):
\end{eulerprompt}
\eulerimg{15}{images/22305141029_EMT4Statistika_Theresia Selvina-043.png}
\eulerheading{Uji}
\begin{eulercomment}
Uji adalah alat penting dalam statistik. Di Euler, banyak uji
diimplementasikan. Semua uji ini mengembalikan kesalahan yang kita
terima jika kita menolak hipotesis nol.

Sebagai contoh, kita menguji lemparan dadu untuk distribusi seragam.
Pada 600 lemparan, kami mendapatkan nilai-nilai berikut, yang kami
masukkan ke uji chi-kuadrat.
\end{eulercomment}
\begin{eulerprompt}
>chitest([90,103,114,101,103,89],dup(100,6)')
\end{eulerprompt}
\begin{euleroutput}
  0.498830517952
\end{euleroutput}
\begin{eulercomment}
Uji chi-kuadrat juga memiliki mode yang menggunakan simulasi Monte
Carlo untuk menguji statistik. Hasilnya seharusnya hampir sama.
Parameter \textgreater{}p mengartikan vektor y sebagai vektor probabilitas.
\end{eulercomment}
\begin{eulerprompt}
>chitest([90,103,114,101,103,89],dup(1/6,6)',>p,>montecarlo)
\end{eulerprompt}
\begin{euleroutput}
  0.526
\end{euleroutput}
\begin{eulercomment}
Kesalahan ini jauh terlalu besar. Jadi kami tidak bisa menolak
distribusi seragam. Ini tidak membuktikan bahwa dadu kami adil. Tetapi
kami tidak bisa menolak hipotesis kami.

Selanjutnya kami menghasilkan 1000 lemparan dadu menggunakan generator
angka acak, dan melakukan uji yang sama.
\end{eulercomment}
\begin{eulerprompt}
>n=1000; t=random([1,n*6]); chitest(count(t*6,6),dup(n,6)')
\end{eulerprompt}
\begin{euleroutput}
  0.528028118442
\end{euleroutput}
\begin{eulercomment}
Mari kita uji untuk nilai rata-rata 100 dengan uji t.
\end{eulercomment}
\begin{eulerprompt}
>s=200+normal([1,100])*10; ...
>ttest(mean(s),dev(s),100,200)
\end{eulerprompt}
\begin{euleroutput}
  0.0218365848476
\end{euleroutput}
\begin{eulercomment}
Fungsi ttest() membutuhkan nilai rata-rata, deviasi, jumlah data, dan
nilai rata-rata yang diuji. 

Sekarang mari kita periksa dua pengukuran untuk rata-rata yang sama.
Kami menolak hipotesis bahwa mereka memiliki rata-rata yang sama, jika
hasilnya \textless{}0,05.
\end{eulercomment}
\begin{eulerprompt}
>tcomparedata(normal(1,10),normal(1,10))
\end{eulerprompt}
\begin{euleroutput}
  0.38722000942
\end{euleroutput}
\begin{eulercomment}
Jika kami menambahkan bias ke salah satu distribusi, kami mendapatkan
lebih banyak penolakan. Ulangi simulasi ini beberapa kali untuk
melihat efeknya.
\end{eulercomment}
\begin{eulerprompt}
>tcomparedata(normal(1,10),normal(1,10)+2)
\end{eulerprompt}
\begin{euleroutput}
  5.60009101758e-07
\end{euleroutput}
\begin{eulercomment}
Dalam contoh berikut, kami menghasilkan 20 lemparan dadu acak 100 kali
dan menghitung jumlah satunya. Harus ada 20/6=3,3 satu rata-rata.
\end{eulercomment}
\begin{eulerprompt}
>R=random(100,20); R=sum(R*6<=1)'; mean(R)
\end{eulerprompt}
\begin{euleroutput}
  3.28
\end{euleroutput}
\begin{eulercomment}
Sekarang kami membandingkan jumlah satu dengan distribusi binomial.
Pertama kami memplot distribusi satu.
\end{eulercomment}
\begin{eulerprompt}
>plot2d(R,distribution=max(R)+1,even=1,style="\(\backslash\)/"):
\end{eulerprompt}
\eulerimg{15}{images/22305141029_EMT4Statistika_Theresia Selvina-044.png}
\begin{eulerprompt}
>t=count(R,21);
\end{eulerprompt}
\begin{eulercomment}
Kemudian kami menghitung nilai yang diharapkan.
\end{eulercomment}
\begin{eulerprompt}
>n=0:20; b=bin(20,n)*(1/6)^n*(5/6)^(20-n)*100;
\end{eulerprompt}
\begin{eulercomment}
Kami harus mengumpulkan beberapa angka untuk mendapatkan kategori yang
cukup besar.
\end{eulercomment}
\begin{eulerprompt}
>t1=sum(t[1:2])|t[3:7]|sum(t[8:21]); ...
>b1=sum(b[1:2])|b[3:7]|sum(b[8:21]);
\end{eulerprompt}
\begin{eulercomment}
Uji chi-kuadrat menolak hipotesis bahwa distribusi kami adalah
distribusi binomial, jika hasilnya \textless{}0,05.
\end{eulercomment}
\begin{eulerprompt}
>chitest(t1,b1)
\end{eulerprompt}
\begin{euleroutput}
  0.53921579764
\end{euleroutput}
\begin{eulercomment}
Contoh berikut berisi hasil dari dua kelompok orang (laki-laki dan
perempuan, katakanlah) yang memilih satu dari enam partai.
\end{eulercomment}
\begin{eulerprompt}
>A=[23,37,43,52,64,74;27,39,41,49,63,76];  ...
>  writetable(A,wc=6,labr=["m","f"],labc=1:6)
\end{eulerprompt}
\begin{euleroutput}
             1     2     3     4     5     6
       m    23    37    43    52    64    74
       f    27    39    41    49    63    76
\end{euleroutput}
\begin{eulercomment}
Kami ingin menguji independensi suara dari jenis kelamin. Uji tabel
chi-kuadrat melakukan ini. Hasilnya terlalu besar untuk menolak
independensi. Jadi kami tidak dapat mengatakan, apakah pemilihan
tergantung pada jenis kelamin dari data ini.
\end{eulercomment}
\begin{eulerprompt}
>tabletest(A)
\end{eulerprompt}
\begin{euleroutput}
  0.990701632326
\end{euleroutput}
\begin{eulercomment}
Berikut adalah tabel yang diharapkan, jika kita mengasumsikan
frekuensi pemilihan yang diamati.
\end{eulercomment}
\begin{eulerprompt}
>writetable(expectedtable(A),wc=6,dc=1,labr=["m","f"],labc=1:6)
\end{eulerprompt}
\begin{euleroutput}
             1     2     3     4     5     6
       m  24.9  37.9  41.9  50.3  63.3  74.7
       f  25.1  38.1  42.1  50.7  63.7  75.3
\end{euleroutput}
\begin{eulercomment}
Kami dapat menghitung koefisien kontingensi yang diperbaiki. Karena
sangat dekat dengan 0, kami menyimpulkan bahwa pemilihan tidak
tergantung pada jenis kelamin.
\end{eulercomment}
\begin{eulerprompt}
>contingency(A)
\end{eulerprompt}
\begin{euleroutput}
  0.0427225484717
\end{euleroutput}
\begin{eulercomment}
\begin{eulercomment}
\eulerheading{Beberapa Uji Lainnya}
\begin{eulercomment}
Selanjutnya kami menggunakan analisis variansi (uji F) untuk menguji
tiga sampel data yang didistribusikan secara normal untuk nilai
rata-rata yang sama. Metodenya disebut ANOVA (analisis variansi). Di
Euler, digunakan fungsi varanalysis().
\end{eulercomment}
\begin{eulerprompt}
>x1=[109,111,98,119,91,118,109,99,115,109,94]; mean(x1),
\end{eulerprompt}
\begin{euleroutput}
  106.545454545
\end{euleroutput}
\begin{eulerprompt}
>x2=[120,124,115,139,114,110,113,120,117]; mean(x2),
\end{eulerprompt}
\begin{euleroutput}
  119.111111111
\end{euleroutput}
\begin{eulerprompt}
>x3=[120,112,115,110,105,134,105,130,121,111]; mean(x3)
\end{eulerprompt}
\begin{euleroutput}
  116.3
\end{euleroutput}
\begin{eulerprompt}
>varanalysis(x1,x2,x3)
\end{eulerprompt}
\begin{euleroutput}
  0.0138048221371
\end{euleroutput}
\begin{eulercomment}
Ini berarti, kita menolak hipotesis nilai rata-rata yang sama. Kami
melakukan ini dengan kemungkinan kesalahan sebesar 1,3\%. 

Ada juga uji median, yang menolak sampel data dengan distribusi
rata-rata yang berbeda yang menguji median dari sampel yang disatukan.
\end{eulercomment}
\begin{eulerprompt}
>a=[56,66,68,49,61,53,45,58,54];
>b=[72,81,51,73,69,78,59,67,65,71,68,71];
>mediantest(a,b)
\end{eulerprompt}
\begin{euleroutput}
  0.0241724220052
\end{euleroutput}
\begin{eulercomment}
Uji lain tentang kesetaraan adalah uji peringkat. Ini jauh lebih tajam
daripada uji median.
\end{eulercomment}
\begin{eulerprompt}
>ranktest(a,b)
\end{eulerprompt}
\begin{euleroutput}
  0.00199969612469
\end{euleroutput}
\begin{eulercomment}
Dalam contoh berikut, kedua distribusi memiliki rata-rata yang sama.
\end{eulercomment}
\begin{eulerprompt}
>ranktest(random(1,100),random(1,50)*3-1)
\end{eulerprompt}
\begin{euleroutput}
  0.129608141484
\end{euleroutput}
\begin{eulercomment}
Mari kita mencoba mensimulasikan dua perlakuan a dan b yang diberikan
kepada orang yang berbeda.
\end{eulercomment}
\begin{eulerprompt}
>a=[8.0,7.4,5.9,9.4,8.6,8.2,7.6,8.1,6.2,8.9];
>b=[6.8,7.1,6.8,8.3,7.9,7.2,7.4,6.8,6.8,8.1];
\end{eulerprompt}
\begin{eulercomment}
Uji signum menentukan apakah a lebih baik daripada b.
\end{eulercomment}
\begin{eulerprompt}
>signtest(a,b)
\end{eulerprompt}
\begin{euleroutput}
  0.0546875
\end{euleroutput}
\begin{eulercomment}
Ini terlalu besar kesalahan. Kami tidak dapat menolak bahwa a sebaik
b.

Uji Wilcoxon lebih tajam daripada uji ini, tetapi mengandalkan nilai
kuantitatif dari perbedaan.
\end{eulercomment}
\begin{eulerprompt}
>wilcoxon(a,b)
\end{eulerprompt}
\begin{euleroutput}
  0.0296680599405
\end{euleroutput}
\begin{eulercomment}
Mari kita mencoba dua uji lain menggunakan rangkaian yang dihasilkan.
\end{eulercomment}
\begin{eulerprompt}
>wilcoxon(normal(1,20),normal(1,20)-1)
\end{eulerprompt}
\begin{euleroutput}
  0.0068706451766
\end{euleroutput}
\begin{eulerprompt}
>wilcoxon(normal(1,20),normal(1,20))
\end{eulerprompt}
\begin{euleroutput}
  0.275145971064
\end{euleroutput}
\eulerheading{Angka Acak}
\begin{eulercomment}
Berikut adalah tes untuk generator angka acak. Euler menggunakan
generator yang sangat baik, sehingga kita tidak perlu mengharapkan
masalah apa pun. 

Pertama kita menghasilkan sepuluh juta angka acak dalam [0,1].
\end{eulercomment}
\begin{eulerprompt}
>n:=10000000; r:=random(1,n);
\end{eulerprompt}
\begin{eulercomment}
Selanjutnya kami menghitung jarak antara dua angka yang kurang dari
0,05.
\end{eulercomment}
\begin{eulerprompt}
>a:=0.05; d:=differences(nonzeros(r<a));
\end{eulerprompt}
\begin{eulercomment}
Terakhir, kami memplot jumlah kali setiap jarak terjadi, dan
membandingkannya dengan nilai yang diharapkan.
\end{eulercomment}
\begin{eulerprompt}
>m=getmultiplicities(1:100,d); plot2d(m); ...
>  plot2d("n*(1-a)^(x-1)*a^2",color=red,>add):
\end{eulerprompt}
\eulerimg{15}{images/22305141029_EMT4Statistika_Theresia Selvina-045.png}
\begin{eulercomment}
Hapus data.
\end{eulercomment}
\begin{eulerprompt}
>remvalue n;
\end{eulerprompt}
\begin{eulercomment}
\begin{eulercomment}
\eulerheading{Pengantar untuk Pengguna Proyek R}
\begin{eulercomment}
elas, EMT tidak bersaing dengan R sebagai paket statistik. Namun, ada
banyak prosedur dan fungsi statistik yang tersedia di EMT juga. Jadi
EMT mungkin memenuhi kebutuhan dasar. Pada akhirnya, EMT dilengkapi
dengan paket-paket numerik dan sistem aljabar komputer.

Buku catatan ini untuk Anda jika Anda akrab dengan R, tetapi perlu
mengetahui perbedaan sintaksis EMT dan R. Kami mencoba memberikan
gambaran tentang hal-hal yang perlu Anda ketahui, baik yang terlihat
maupun yang kurang terlihat. 

Selain itu, kami akan melihat cara pertukaran data antara kedua sistem
tersebut.
\end{eulercomment}
\begin{eulercomment}
Perhatikan bahwa ini masih dalam tahap pengembangan.
\end{eulercomment}
\eulerheading{Basic Syntax}
\begin{eulercomment}
Hal pertama yang Anda pelajari di R adalah membuat vektor. Di EMT,
perbedaan utamanya adalah operator ":" dapat mengambil langkah. Selain
itu, ini memiliki kekuatan ikatan yang rendah.
\end{eulercomment}
\begin{eulerprompt}
>n=10; 0:n/20:n-1
\end{eulerprompt}
\begin{euleroutput}
  [0,  0.5,  1,  1.5,  2,  2.5,  3,  3.5,  4,  4.5,  5,  5.5,  6,  6.5,
  7,  7.5,  8,  8.5,  9]
\end{euleroutput}
\begin{eulercomment}
Fungsi c() tidak ada. Memungkinkan untuk menggunakan vektor untuk
menggabungkan hal-hal. 

Contoh berikut adalah, seperti banyak lainnya, dari "Pengenalan ke R"
yang disertakan dalam proyek R. Jika Anda membaca PDF ini, Anda akan
menemukan bahwa saya mengikuti jalannya dalam tutorial ini.
\end{eulercomment}
\begin{eulerprompt}
>x=[10.4, 5.6, 3.1, 6.4, 21.7]; [x,0,x]
\end{eulerprompt}
\begin{euleroutput}
  [10.4,  5.6,  3.1,  6.4,  21.7,  0,  10.4,  5.6,  3.1,  6.4,  21.7]
\end{euleroutput}
\begin{eulercomment}
Operator titik dua dengan langkah EMT digantikan oleh fungsi seq() di
R. Kita bisa menulis fungsi ini dalam EMT.
\end{eulercomment}
\begin{eulerprompt}
>function seq(a,b,c) := a:b:c; ...
>seq(0,-0.1,-1)
\end{eulerprompt}
\begin{euleroutput}
  [0,  -0.1,  -0.2,  -0.3,  -0.4,  -0.5,  -0.6,  -0.7,  -0.8,  -0.9,  -1]
\end{euleroutput}
\begin{eulercomment}
Fungsi rep() R tidak ada dalam EMT. Untuk input vektor, bisa ditulis
seperti berikut.
\end{eulercomment}
\begin{eulerprompt}
>function rep(x:vector,n:index) := flatten(dup(x,n)); ...
>rep(x,2)
\end{eulerprompt}
\begin{euleroutput}
  [10.4,  5.6,  3.1,  6.4,  21.7,  10.4,  5.6,  3.1,  6.4,  21.7]
\end{euleroutput}
\begin{eulercomment}
Perhatikan bahwa "=" atau ":=" digunakan untuk penugasan. Operator
"-\textgreater{}" digunakan untuk satuan di EMT.
\end{eulercomment}
\begin{eulerprompt}
>125km -> " miles"
\end{eulerprompt}
\begin{euleroutput}
  77.6713990297 miles
\end{euleroutput}
\begin{eulercomment}
Operator "\textless{}-" untuk penugasan adalah membingungkan, dan bukan ide yang
baik dalam R. Berikut akan membandingkan a dan -4 di EMT.
\end{eulercomment}
\begin{eulerprompt}
>a=2; a<-4
\end{eulerprompt}
\begin{euleroutput}
  0
\end{euleroutput}
\begin{eulercomment}
Di R, "a\textless{}-4\textless{}3" " berfungsi, tetapi "a\textless{}-4\textless{}-3" tidak. Saya memiliki
ambiguitas serupa di EMT juga, tetapi mencoba untuk menghilangkannya
perlahan-lahan.

EMT dan R memiliki vektor tipe boolean. Tetapi di EMT, angka 0 dan 1
digunakan untuk mewakili false dan true. Di R, nilai true dan false
dapat digunakan dalam aritmatika biasa seperti di EMT.
\end{eulercomment}
\begin{eulerprompt}
>x<5, %*x
\end{eulerprompt}
\begin{euleroutput}
  [0,  0,  1,  0,  0]
  [0,  0,  3.1,  0,  0]
\end{euleroutput}
\begin{eulercomment}
EMT menghasilkan kesalahan atau menghasilkan NAN tergantung pada flag
"errors".
\end{eulercomment}
\begin{eulerprompt}
>errors off; 0/0, isNAN(sqrt(-1)), errors on;
\end{eulerprompt}
\begin{euleroutput}
  NAN
  1
\end{euleroutput}
\begin{eulercomment}
String sama di R dan EMT. Keduanya dalam bahasa lokal saat ini, bukan
dalam Unicode. 

Di R ada paket untuk Unicode. Di EMT, string dapat menjadi string
Unicode. String Unicode dapat diterjemahkan ke encoding lokal dan
sebaliknya. Selain itu, u"..." dapat berisi entitas HTML.
\end{eulercomment}
\begin{eulerprompt}
>u"&#169; Ren&eacut; Grothmann"
\end{eulerprompt}
\begin{euleroutput}
  © René Grothmann
\end{euleroutput}
\begin{eulercomment}
Berikut mungkin akan atau tidak akan tampil dengan benar di sistem
Anda sebagai A dengan titik dan garis atas. Itu tergantung pada font
yang Anda gunakan.
\end{eulercomment}
\begin{eulerprompt}
>chartoutf([480])
\end{eulerprompt}
\begin{euleroutput}
  Ǡ
\end{euleroutput}
\begin{eulercomment}
Penggabungan string dilakukan dengan "+" atau "\textbar{}". Ini dapat mencakup
angka, yang akan dicetak dalam format saat ini.
\end{eulercomment}
\begin{eulerprompt}
>"pi = "+pi
\end{eulerprompt}
\begin{euleroutput}
  pi = 3.14159265359
\end{euleroutput}
\eulerheading{Indexing}
\begin{eulercomment}
Sebagian besar waktu, ini akan berfungsi seperti di R. 

Tetapi EMT akan menginterpretasikan indeks negatif dari belakang
vektor, sementara R menginterpretasikan x[n] sebagai x tanpa elemen
ke-n.
\end{eulercomment}
\begin{eulerprompt}
>x, x[1:3], x[-2]
\end{eulerprompt}
\begin{euleroutput}
  [10.4,  5.6,  3.1,  6.4,  21.7]
  [10.4,  5.6,  3.1]
  6.4
\end{euleroutput}
\begin{eulercomment}
Perilaku R bisa dicapai di EMT dengan drop().
\end{eulercomment}
\begin{eulerprompt}
>drop(x,2)
\end{eulerprompt}
\begin{euleroutput}
  [10.4,  3.1,  6.4,  21.7]
\end{euleroutput}
\begin{eulercomment}
Vektor logika tidak diperlakukan secara berbeda sebagai indeks di EMT,
berbeda dengan R. Anda perlu mengekstrak elemen yang tidak nol
terlebih dahulu di EMT.
\end{eulercomment}
\begin{eulerprompt}
>x, x>5, x[nonzeros(x>5)]
\end{eulerprompt}
\begin{euleroutput}
  [10.4,  5.6,  3.1,  6.4,  21.7]
  [1,  1,  0,  1,  1]
  [10.4,  5.6,  6.4,  21.7]
\end{euleroutput}
\begin{eulercomment}
Seperti di R, vektor indeks dapat berisi pengulangan.
\end{eulercomment}
\begin{eulerprompt}
>x[[1,2,2,1]]
\end{eulerprompt}
\begin{euleroutput}
  [10.4,  5.6,  5.6,  10.4]
\end{euleroutput}
\begin{eulercomment}
Tetapi nama untuk indeks tidak mungkin di EMT. Untuk paket statistik,
ini seringkali diperlukan untuk memudahkan akses ke elemen-elemen
vektor. 

Untuk meniru perilaku ini, kita bisa mendefinisikan fungsi seperti
berikut.
\end{eulercomment}
\begin{eulerprompt}
>function sel (v,i,s) := v[indexof(s,i)]; ...
>s=["first","second","third","fourth"]; sel(x,["first","third"],s)
\end{eulerprompt}
\begin{euleroutput}
  
  Trying to overwrite protected function sel!
  Error in:
  function sel (v,i,s) := v[indexof(s,i)]; ... ...
               ^
  [10.4,  3.1]
\end{euleroutput}
\eulerheading{Data Types}
\begin{eulercomment}
EMT memiliki lebih banyak tipe data yang tetap daripada R. Jelas, di R
ada vektor yang berkembang. Anda bisa mengatur vektor numerik kosong v
dan memberikan nilai pada elemen v[17]. Ini tidak mungkin di EMT.

Berikut adalah sedikit tidak efisien.
\end{eulercomment}
\begin{eulerprompt}
>v=[]; for i=1 to 10000; v=v|i; end;
\end{eulerprompt}
\begin{eulercomment}
EMT sekarang akan membuat vektor dengan v dan i ditambahkan ke
tumpukan dan menyalin kembali vektor itu ke variabel global v. 

Yang lebih efisien lebih didefinisikan sebelumnya.
\end{eulercomment}
\begin{eulerprompt}
>v=zeros(10000); for i=1 to 10000; v[i]=i; end;
\end{eulerprompt}
\begin{eulercomment}
Untuk mengubah tipe data di EMT, Anda bisa menggunakan fungsi seperti
complex().
\end{eulercomment}
\begin{eulerprompt}
>complex(1:4)
\end{eulerprompt}
\begin{euleroutput}
  [ 1+0i ,  2+0i ,  3+0i ,  4+0i  ]
\end{euleroutput}
\begin{eulercomment}
Konversi menjadi string hanya mungkin untuk tipe data dasar. Format
saat ini digunakan untuk penggabungan string sederhana. Tetapi ada
fungsi seperti print() atau frac(). 

Untuk vektor, Anda bisa dengan mudah menulis fungsi Anda sendiri.
\end{eulercomment}
\begin{eulerprompt}
>function tostr (v) ...
\end{eulerprompt}
\begin{eulerudf}
  s="[";
  loop 1 to length(v);
     s=s+print(v[#],2,0);
     if #<length(v) then s=s+","; endif;
  end;
  return s+"]";
  endfunction
\end{eulerudf}
\begin{eulerprompt}
>tostr(linspace(0,1,10))
\end{eulerprompt}
\begin{euleroutput}
  [0.00,0.10,0.20,0.30,0.40,0.50,0.60,0.70,0.80,0.90,1.00]
\end{euleroutput}
\begin{eulercomment}
Untuk berkomunikasi dengan Maxima, ada fungsi convertmxm(), yang juga
dapat digunakan untuk memformat vektor untuk output.
\end{eulercomment}
\begin{eulerprompt}
>convertmxm(1:10)
\end{eulerprompt}
\begin{euleroutput}
  [1,2,3,4,5,6,7,8,9,10]
\end{euleroutput}
\begin{eulercomment}
Untuk Latex, perintah tex dapat digunakan untuk mendapatkan perintah
Latex.
\end{eulercomment}
\begin{eulerprompt}
>tex(&[1,2,3])
\end{eulerprompt}
\begin{euleroutput}
  \(\backslash\)left[ 1 , 2 , 3 \(\backslash\)right] 
\end{euleroutput}
\eulerheading{Factors and Tables}
\begin{eulercomment}
Dalam pengenalan ke R ada contoh dengan faktor yang disebut faktor.

Berikut adalah daftar wilayah dari 30 negara bagian.
\end{eulercomment}
\begin{eulerprompt}
>austates = ["tas", "sa", "qld", "nsw", "nsw", "nt", "wa", "wa", ...
>"qld", "vic", "nsw", "vic", "qld", "qld", "sa", "tas", ...
>"sa", "nt", "wa", "vic", "qld", "nsw", "nsw", "wa", ...
>"sa", "act", "nsw", "vic", "vic", "act"];
\end{eulerprompt}
\begin{eulercomment}
Misalkan, kami memiliki pendapatan yang sesuai di setiap negara
bagian.
\end{eulercomment}
\begin{eulerprompt}
>incomes = [60, 49, 40, 61, 64, 60, 59, 54, 62, 69, 70, 42, 56, ...
>61, 61, 61, 58, 51, 48, 65, 49, 49, 41, 48, 52, 46, ...
>59, 46, 58, 43];
\end{eulerprompt}
\begin{eulercomment}
Sekarang, kami ingin menghitung rata-rata pendapatan di wilayah.
Sebagai program statistik, R memiliki faktor() dan tapply() untuk ini.

EMT dapat melakukannya dengan menemukan indeks wilayah dalam daftar
wilayah unik.
\end{eulercomment}
\begin{eulerprompt}
>auterr=sort(unique(austates)); f=indexofsorted(auterr,austates)
\end{eulerprompt}
\begin{euleroutput}
  [6,  5,  4,  2,  2,  3,  8,  8,  4,  7,  2,  7,  4,  4,  5,  6,  5,  3,
  8,  7,  4,  2,  2,  8,  5,  1,  2,  7,  7,  1]
\end{euleroutput}
\begin{eulercomment}
Pada saat itu, kami bisa menulis fungsi loop kami sendiri untuk
melakukan hal-hal untuk satu faktor saja. 

Atau kita bisa meniru fungsi tapply() dengan cara berikut.
\end{eulercomment}
\begin{eulerprompt}
>function map tappl (i; f$:call, cat, x) ...
\end{eulerprompt}
\begin{eulerudf}
  u=sort(unique(cat));
  f=indexof(u,cat);
  return f$(x[nonzeros(f==indexof(u,i))]);
  endfunction
\end{eulerudf}
\begin{eulercomment}
Ini agak tidak efisien, karena menghitung wilayah unik untuk setiap i,
tetapi ini berfungsi.
\end{eulercomment}
\begin{eulerprompt}
>tappl(auterr,"mean",austates,incomes)
\end{eulerprompt}
\begin{euleroutput}
  [44.5,  57.3333333333,  55.5,  53.6,  55,  60.5,  56,  52.25]
\end{euleroutput}
\begin{eulercomment}
Perhatikan bahwa ini berfungsi untuk setiap vektor wilayah.
\end{eulercomment}
\begin{eulerprompt}
>tappl(["act","nsw"],"mean",austates,incomes)
\end{eulerprompt}
\begin{euleroutput}
  [44.5,  57.3333333333]
\end{euleroutput}
\begin{eulercomment}
Sekarang, paket statistik EMT mendefinisikan tabel seperti di R.
Fungsi readtable() dan writetable() dapat digunakan untuk input dan
output.

Jadi kita bisa mencetak rata-rata pendapatan negara bagian dalam
wilayah dengan cara yang ramah.
\end{eulercomment}
\begin{eulerprompt}
>writetable(tappl(auterr,"mean",austates,incomes),labc=auterr,wc=7)
\end{eulerprompt}
\begin{euleroutput}
      act    nsw     nt    qld     sa    tas    vic     wa
     44.5  57.33   55.5   53.6     55   60.5     56  52.25
\end{euleroutput}
\begin{eulercomment}
Kita juga bisa mencoba meniru perilaku R sepenuhnya.

Faktor jelas harus tetap dalam koleksi dengan tipe dan kategori
(negara bagian dan wilayah dalam contoh kami). Untuk EMT, kami
menambahkan indeks yang telah dihitung sebelumnya.
\end{eulercomment}
\begin{eulerprompt}
>function makef (t) ...
\end{eulerprompt}
\begin{eulerudf}
  ## Factor data
  ## Returns a collection with data t, unique data, indices.
  ## See: tapply
  u=sort(unique(t));
  return \{\{t,u,indexofsorted(u,t)\}\};
  endfunction
\end{eulerudf}
\begin{eulerprompt}
>statef=makef(austates);
\end{eulerprompt}
\begin{eulercomment}
Sekarang elemen ketiga koleksi akan berisi indeks.
\end{eulercomment}
\begin{eulerprompt}
>statef[3]
\end{eulerprompt}
\begin{euleroutput}
  [6,  5,  4,  2,  2,  3,  8,  8,  4,  7,  2,  7,  4,  4,  5,  6,  5,  3,
  8,  7,  4,  2,  2,  8,  5,  1,  2,  7,  7,  1]
\end{euleroutput}
\begin{eulercomment}
Sekarang kita bisa meniru tapply() dengan cara berikut. Ini akan
mengembalikan tabel sebagai koleksi data tabel dan judul kolom.
\end{eulercomment}
\begin{eulerprompt}
>function tapply (t:vector,tf,f$:call) ...
\end{eulerprompt}
\begin{eulerudf}
  ## Makes a table of data and factors
  ## tf : output of makef()
  ## See: makef
  uf=tf[2]; f=tf[3]; x=zeros(length(uf));
  for i=1 to length(uf);
     ind=nonzeros(f==i);
     if length(ind)==0 then x[i]=NAN;
     else x[i]=f$(t[ind]);
     endif;
  end;
  return \{\{x,uf\}\};
  endfunction
\end{eulerudf}
\begin{eulercomment}
Kami tidak menambahkan banyak pemeriksaan tipe di sini. Satu-satunya
tindakan pencegahan berkaitan dengan kategori (faktor) tanpa data.
Tetapi Anda harus memeriksa panjang t yang benar dan kebenaran koleksi
tf.

Tabel ini dapat dicetak sebagai tabel dengan writetable().
\end{eulercomment}
\begin{eulerprompt}
>writetable(tapply(incomes,statef,"mean"),wc=7)
\end{eulerprompt}
\begin{euleroutput}
      act    nsw     nt    qld     sa    tas    vic     wa
     44.5  57.33   55.5   53.6     55   60.5     56  52.25
\end{euleroutput}
\eulerheading{Arrays}
\begin{eulercomment}
EMT hanya memiliki dua dimensi untuk array. Tipe data disebut matriks.
Akan mudah untuk menulis fungsi untuk dimensi lebih tinggi atau
perpustakaan C untuk ini, bagaimanapun.

R memiliki lebih dari dua dimensi. Dalam R, array adalah vektor dengan
bidang dimensi.

Dalam EMT, vektor adalah matriks dengan satu baris. Ini dapat diubah
menjadi matriks dengan redim().
\end{eulercomment}
\begin{eulerprompt}
>shortformat; X=redim(1:20,4,5)
\end{eulerprompt}
\begin{euleroutput}
          1         2         3         4         5 
          6         7         8         9        10 
         11        12        13        14        15 
         16        17        18        19        20 
\end{euleroutput}
\begin{eulercomment}
Pengambilan baris dan kolom, atau sub-matriks, mirip dengan di R.
\end{eulercomment}
\begin{eulerprompt}
>X[,2:3]
\end{eulerprompt}
\begin{euleroutput}
          2         3 
          7         8 
         12        13 
         17        18 
\end{euleroutput}
\begin{eulercomment}
Namun, di R mungkin untuk mengatur daftar indeks tertentu dari vektor
ke nilai. Hal yang sama hanya mungkin di EMT dengan loop.
\end{eulercomment}
\begin{eulerprompt}
>function setmatrixvalue (M, i, j, v) ...
\end{eulerprompt}
\begin{eulerudf}
  loop 1 to max(length(i),length(j),length(v))
     M[i\{#\},j\{#\}] = v\{#\};
  end;
  endfunction
\end{eulerudf}
\begin{eulercomment}
Kami mendemonstrasikannya untuk menunjukkan bahwa matriks diteruskan
dengan referensi di EMT. Jika Anda tidak ingin mengubah matriks asli
M, Anda perlu menyalinnya dalam fungsi.
\end{eulercomment}
\begin{eulerprompt}
>setmatrixvalue(X,1:3,3:-1:1,0); X,
\end{eulerprompt}
\begin{euleroutput}
          1         2         0         4         5 
          6         0         8         9        10 
          0        12        13        14        15 
         16        17        18        19        20 
\end{euleroutput}
\begin{eulercomment}
Perkalian luar di EMT hanya bisa dilakukan antara vektor. Ini otomatis
karena bahasa matriks. Salah satu vektor harus menjadi vektor kolom
dan yang lainnya vektor baris.
\end{eulercomment}
\begin{eulerprompt}
>(1:5)*(1:5)'
\end{eulerprompt}
\begin{euleroutput}
          1         2         3         4         5 
          2         4         6         8        10 
          3         6         9        12        15 
          4         8        12        16        20 
          5        10        15        20        25 
\end{euleroutput}
\begin{eulercomment}
Dalam pengantar PDF untuk R ada contoh, yang menghitung distribusi
ab-cd untuk a, b, c, d yang dipilih dari 0 hingga n secara acak.
Solusi di R adalah bentuk matriks berdimensi 4 dan menjalankan table()
di atasnya.

Tentu saja, ini bisa dicapai dengan loop. Tetapi loop tidak efektif
dalam EMT atau R. Dalam EMT, kita bisa menulis loop dalam C dan itu
akan menjadi solusi yang paling cepat.

Tetapi kami ingin meniru perilaku R. Untuk ini, kita perlu meratakan
perkalian ab dan membuat matriks ab-cd.
\end{eulercomment}
\begin{eulerprompt}
>a=0:6; b=a'; p=flatten(a*b); q=flatten(p-p'); ...
>u=sort(unique(q)); f=getmultiplicities(u,q); ...
>statplot(u,f,"h"):
\end{eulerprompt}
\eulerimg{15}{images/22305141029_EMT4Statistika_Theresia Selvina-046.png}
\begin{eulercomment}
Selain kelipatan yang tepat, EMT dapat menghitung frekuensi dalam
vektor.
\end{eulercomment}
\begin{eulerprompt}
>getfrequencies(q,-50:10:50)
\end{eulerprompt}
\begin{euleroutput}
  [0,  23,  132,  316,  602,  801,  333,  141,  53,  0]
\end{euleroutput}
\begin{eulercomment}
Cara paling mudah untuk memplot ini sebagai distribusi adalah sebagai
berikut.
\end{eulercomment}
\begin{eulerprompt}
>plot2d(q,distribution=11):
\end{eulerprompt}
\eulerimg{15}{images/22305141029_EMT4Statistika_Theresia Selvina-047.png}
\begin{eulercomment}
Tetapi juga mungkin untuk memprediksi jumlah dalam interval yang telah
dipilih sebelumnya. Tentu saja, yang berikut ini menggunakan
getfrequencies() secara internal.

Karena fungsi histo() mengembalikan frekuensi, kami perlu menskalakan
frekuensi ini sehingga integral di bawah grafik batang adalah 1.
\end{eulercomment}
\begin{eulerprompt}
>\{x,y\}=histo(q,v=-55:10:55); y=y/sum(y)/differences(x); ...
>plot2d(x,y,>bar,style="/"):
\end{eulerprompt}
\eulerimg{15}{images/22305141029_EMT4Statistika_Theresia Selvina-048.png}
\eulerheading{Lists}
\begin{eulercomment}
EMT memiliki dua jenis daftar. Satu adalah daftar global yang dapat
diubah, dan yang lain adalah tipe daftar yang tidak dapat diubah. Kami
tidak memperhatikan daftar global di sini. 

Tipe daftar yang tidak dapat diubah disebut koleksi dalam EMT. Ini
berperilaku seperti struktur dalam C, tetapi elemennya hanya dinomori
dan tidak dinamai.
\end{eulercomment}
\begin{eulerprompt}
>L=\{\{"Fred","Flintstone",40,[1990,1992]\}\}
\end{eulerprompt}
\begin{euleroutput}
  Fred
  Flintstone
  40
  [1990,  1992]
\end{euleroutput}
\begin{eulercomment}
Saat ini, elemen-elemen tidak memiliki nama, meskipun nama dapat
diatur untuk tujuan khusus. Mereka diakses dengan nomor.
\end{eulercomment}
\begin{eulerprompt}
>(L[4])[2]
\end{eulerprompt}
\begin{euleroutput}
  1992
\end{euleroutput}
\begin{eulercomment}
\begin{eulercomment}
\eulerheading{Input dan Output File (Membaca dan Menulis Data) }
\begin{eulercomment}
Anda sering ingin mengimpor matriks data dari sumber lain ke EMT.
Tutorial ini memberi tahu Anda tentang banyak cara untuk mencapainya.
Fungsi sederhana adalah writematrix() dan readmatrix(). 

Mari kita tunjukkan bagaimana cara membaca dan menulis vektor bilangan
riil ke file.
\end{eulercomment}
\begin{eulerprompt}
>a=random(1,100); mean(a), dev(a),
\end{eulerprompt}
\begin{euleroutput}
  0.49815
  0.28037
\end{euleroutput}
\begin{eulercomment}
Untuk menulis data ke file, kita menggunakan fungsi writematrix(). 

Karena pengantar ini kemungkinan besar berada dalam direktori di mana
pengguna tidak memiliki hak menulis, kita menulis data ke direktori
rumah pengguna. Untuk notebook Anda sendiri, ini tidak perlu, karena
file data akan ditulis ke direktori yang sama.
\end{eulercomment}
\begin{eulerprompt}
>filename="test.dat";
\end{eulerprompt}
\begin{eulercomment}
Sekarang kita menulis vektor kolom a' ke file. Ini menghasilkan satu
angka dalam setiap baris file.
\end{eulercomment}
\begin{eulerprompt}
>writematrix(a',filename);
\end{eulerprompt}
\begin{eulercomment}
Untuk membaca data, kita gunakan readmatrix().
\end{eulercomment}
\begin{eulerprompt}
>a=readmatrix(filename)';
\end{eulerprompt}
\begin{eulercomment}
Dan hapus file tersebut.
\end{eulercomment}
\begin{eulerprompt}
>fileremove(filename);
>mean(a), dev(a),
\end{eulerprompt}
\begin{euleroutput}
  0.49815
  0.28037
\end{euleroutput}
\begin{eulercomment}
Fungsi writematrix() atau writetable() dapat dikonfigurasi untuk
bahasa lain. 

Misalnya, jika Anda memiliki sistem berbahasa Indonesia (tanda desimal
dengan koma), Excel Anda memerlukan nilai dengan koma desimal yang
dipisahkan oleh titik koma dalam file csv (defaultnya adalah nilai
yang dipisahkan oleh koma). File berikut, "test.csv," harus muncul di
folder saat ini.
\end{eulercomment}
\begin{eulerprompt}
>filename="test.csv"; ...
>writematrix(random(5,3),file=filename,separator=",");
\end{eulerprompt}
\begin{eulercomment}
Anda sekarang dapat membuka file ini dengan Excel berbahasa Indonesia
secara langsung.
\end{eulercomment}
\begin{eulerprompt}
>fileremove(filename);
\end{eulerprompt}
\begin{eulercomment}
Terkadang kita memiliki string dengan token seperti berikut.
\end{eulercomment}
\begin{eulerprompt}
>s1:="f m m f m m m f f f m m f";  ...
>s2:="f f f m m f f";
\end{eulerprompt}
\begin{eulercomment}
Untuk membagi string ini, kita mendefinisikan vektor token.
\end{eulercomment}
\begin{eulerprompt}
>tok:=["f","m"]
\end{eulerprompt}
\begin{euleroutput}
  f
  m
\end{euleroutput}
\begin{eulercomment}
Kemudian kita dapat menghitung jumlah kemunculan masing-masing token
dalam string tersebut, dan menempatkan hasilnya ke dalam sebuah tabel.
\end{eulercomment}
\begin{eulerprompt}
>M:=getmultiplicities(tok,strtokens(s1))_ ...
>  getmultiplicities(tok,strtokens(s2));
\end{eulerprompt}
\begin{eulercomment}
Tulis tabel dengan header token.
\end{eulercomment}
\begin{eulerprompt}
>writetable(M,labc=tok,labr=1:2,wc=8)
\end{eulerprompt}
\begin{euleroutput}
                 f       m
         1       6       7
         2       5       2
\end{euleroutput}
\begin{eulercomment}
Untuk statistik, EMT dapat membaca dan menulis tabel.
\end{eulercomment}
\begin{eulerprompt}
>file="test.dat"; open(file,"w"); ...
>writeln("A,B,C"); writematrix(random(3,3)); ...
>close();
\end{eulerprompt}
\begin{eulercomment}
File terlihat seperti ini.
\end{eulercomment}
\begin{eulerprompt}
>printfile(file)
\end{eulerprompt}
\begin{euleroutput}
  A,B,C
  0.7003664386138074,0.1875530821001213,0.3262339279660414
  0.5926249243193858,0.1522927283984059,0.368140583062521
  0.8065535209872989,0.7265910840408142,0.7332619844597152
  
\end{euleroutput}
\begin{eulercomment}
Fungsi readtable() dalam bentuk yang paling sederhana dapat membaca
ini dan mengembalikan koleksi nilai dan baris judul.
\end{eulercomment}
\begin{eulerprompt}
>L=readtable(file,>list);
\end{eulerprompt}
\begin{eulercomment}
Koleksi ini dapat dicetak dengan writetable() ke notebook, atau ke
file.
\end{eulercomment}
\begin{eulerprompt}
>writetable(L,wc=10,dc=5)
\end{eulerprompt}
\begin{euleroutput}
           A         B         C
     0.70037   0.18755   0.32623
     0.59262   0.15229   0.36814
     0.80655   0.72659   0.73326
\end{euleroutput}
\begin{eulercomment}
Matriks nilai adalah elemen pertama dari L. Perhatikan bahwa mean()
dalam EMT menghitung nilai rata-rata dari baris matriks.
\end{eulercomment}
\begin{eulerprompt}
>mean(L[1])
\end{eulerprompt}
\begin{euleroutput}
    0.40472 
    0.37102 
    0.75547 
\end{euleroutput}
\eulerheading{File CSV}
\begin{eulercomment}
Pertama, mari tulis matriks ke dalam file. Untuk output, kita
menghasilkan file dalam direktori kerja saat ini.
\end{eulercomment}
\begin{eulerprompt}
>file="test.csv";  ...
>M=random(3,3); writematrix(M,file);
\end{eulerprompt}
\begin{eulercomment}
Berikut ini adalah isi dari file tersebut.
\end{eulercomment}
\begin{eulerprompt}
>printfile(file)
\end{eulerprompt}
\begin{euleroutput}
  0.8221197733097619,0.821531098722547,0.7771240608094004
  0.8482947121863489,0.3237767724883862,0.6501422353377985
  0.1482301827518109,0.3297459716109594,0.6261901074210923
  
\end{euleroutput}
\begin{eulercomment}
CSV ini dapat dibuka pada sistem berbahasa Inggris ke Excel dengan
mengklik dua kali. Jika Anda mendapatkan file semacam ini di sistem
berbahasa Jerman, Anda perlu mengimpor data ke Excel dengan
memperhatikan titik desimal. 

Namun titik desimal adalah format default untuk EMT juga. Anda dapat
membaca matriks dari file dengan readmatrix().
\end{eulercomment}
\begin{eulerprompt}
>readmatrix(file)
\end{eulerprompt}
\begin{euleroutput}
    0.82212   0.82153   0.77712 
    0.84829   0.32378   0.65014 
    0.14823   0.32975   0.62619 
\end{euleroutput}
\begin{eulercomment}
Mungkin kita ingin menulis beberapa matriks ke dalam satu file.
Perintah open() dapat membuka file untuk penulisan dengan parameter
"w". Defaultnya adalah "r" untuk membaca.
\end{eulercomment}
\begin{eulerprompt}
>open(file,"w"); writematrix(M); writematrix(M'); close();
\end{eulerprompt}
\begin{eulercomment}
Matriks dipisahkan oleh baris kosong. Untuk membaca matriks, buka file
dan panggil readmatrix() beberapa kali.
\end{eulercomment}
\begin{eulerprompt}
>open(file); A=readmatrix(); B=readmatrix(); A==B, close();
\end{eulerprompt}
\begin{euleroutput}
          1         0         0 
          0         1         0 
          0         0         1 
\end{euleroutput}
\begin{eulercomment}
Di Excel atau spreadsheet serupa, Anda dapat mengekspor matriks
sebagai CSV (nilai yang dipisahkan oleh koma). Di Excel 2007, gunakan
"simpan sebagai" dan "format lain," lalu pilih "CSV". Pastikan tabel
saat ini hanya berisi data yang ingin Anda ekspor. 

Berikut contohnya.
\end{eulercomment}
\begin{eulerprompt}
>printfile("excel-data.csv")
\end{eulerprompt}
\begin{euleroutput}
  0;1000;1000
  1;1051,271096;1072,508181
  2;1105,170918;1150,273799
  3;1161,834243;1233,67806
  4;1221,402758;1323,129812
  5;1284,025417;1419,067549
  6;1349,858808;1521,961556
  7;1419,067549;1632,31622
  8;1491,824698;1750,6725
  9;1568,312185;1877,610579
  10;1648,721271;2013,752707
\end{euleroutput}
\begin{eulercomment}
Seperti yang bisa Anda lihat, sistem berbahasa Jerman saya telah
menggunakan titik koma sebagai pemisah dan koma desimal. Anda dapat
mengubah ini di pengaturan sistem atau di Excel, tetapi itu tidak
diperlukan untuk membaca matriks ke EMT. 

Cara paling mudah untuk membaca ini ke Euler adalah dengan menggunakan
readmatrix(). Semua koma digantikan oleh titik dengan parameter
\textgreater{}comma. Untuk CSV berbahasa Inggris, cukup hilangkan parameter ini.
\end{eulercomment}
\begin{eulerprompt}
>M=readmatrix("excel-data.csv",>comma)
\end{eulerprompt}
\begin{euleroutput}
          0      1000      1000 
          1    1051.3    1072.5 
          2    1105.2    1150.3 
          3    1161.8    1233.7 
          4    1221.4    1323.1 
          5      1284    1419.1 
          6    1349.9      1522 
          7    1419.1    1632.3 
          8    1491.8    1750.7 
          9    1568.3    1877.6 
         10    1648.7    2013.8 
\end{euleroutput}
\begin{eulercomment}
Mari kita gambarkan ini.
\end{eulercomment}
\begin{eulerprompt}
>plot2d(M'[1],M'[2:3],>points,color=[red,green]'):
\end{eulerprompt}
\eulerimg{15}{images/22305141029_EMT4Statistika_Theresia Selvina-049.png}
\begin{eulercomment}
Ada cara yang lebih sederhana untuk membaca data dari file. Anda dapat
membuka file dan membaca angka-angka baris demi baris. Fungsi
getvectorline() akan membaca angka dari baris data. Secara default, ia
mengharapkan titik desimal. Tetapi ia juga dapat menggunakan koma
desimal, jika Anda memanggil setdecimaldot(",") sebelum Anda
menggunakan fungsi ini. 

Fungsi berikut adalah contoh untuk ini. Ia akan berhenti di akhir file
atau baris kosong.
\end{eulercomment}
\begin{eulerprompt}
>function myload (file) ...
\end{eulerprompt}
\begin{eulerudf}
  open(file);
  M=[];
  repeat
     until eof();
     v=getvectorline(3);
     if length(v)>0 then M=M_v; else break; endif;
  end;
  return M;
  close(file);
  endfunction
\end{eulerudf}
\begin{eulerprompt}
>myload(file)
\end{eulerprompt}
\begin{euleroutput}
    0.82212   0.82153   0.77712 
    0.84829   0.32378   0.65014 
    0.14823   0.32975   0.62619 
\end{euleroutput}
\begin{eulercomment}
Juga mungkin membaca semua angka dalam file dengan getvector().
\end{eulercomment}
\begin{eulerprompt}
>open(file); v=getvector(10000); close(); redim(v[1:9],3,3)
\end{eulerprompt}
\begin{euleroutput}
    0.82212   0.82153   0.77712 
    0.84829   0.32378   0.65014 
    0.14823   0.32975   0.62619 
\end{euleroutput}
\begin{eulercomment}
Jadi sangat mudah untuk menyimpan vektor nilai, satu nilai dalam
setiap baris, dan membacanya kembali.
\end{eulercomment}
\begin{eulerprompt}
>v=random(1000); mean(v)
\end{eulerprompt}
\begin{euleroutput}
  0.50303
\end{euleroutput}
\begin{eulerprompt}
>writematrix(v',file); mean(readmatrix(file)')
\end{eulerprompt}
\begin{euleroutput}
  0.50303
\end{euleroutput}
\eulerheading{Menggunakan Tabel }
\begin{eulercomment}
Tabel dapat digunakan untuk membaca atau menulis data numerik. Sebagai
contoh, kita menulis tabel dengan baris dan judul kolom ke dalam file.
\end{eulercomment}
\begin{eulerprompt}
>file="test.tab"; M=random(3,3);  ...
>open(file,"w");  ...
>writetable(M,separator=",",labc=["one","two","three"]);  ...
>close(); ...
>printfile(file)
\end{eulerprompt}
\begin{euleroutput}
  one,two,three
        0.09,      0.39,      0.86
        0.39,      0.86,      0.71
         0.2,      0.02,      0.83
\end{euleroutput}
\begin{eulercomment}
Ini dapat diimpor ke Excel. 

Untuk membaca file dalam EMT, kita menggunakan readtable().
\end{eulercomment}
\begin{eulerprompt}
>\{M,headings\}=readtable(file,>clabs); ...
>writetable(M,labc=headings)
\end{eulerprompt}
\begin{euleroutput}
         one       two     three
        0.09      0.39      0.86
        0.39      0.86      0.71
         0.2      0.02      0.83
\end{euleroutput}
\eulerheading{Menganalisis Sebuah Baris }
\begin{eulercomment}
Anda bahkan bisa mengevaluasi setiap baris secara manual. Misalkan
kita memiliki baris dengan format berikut.
\end{eulercomment}
\begin{eulerprompt}
>line="2020-11-03,Tue,1'114.05"
\end{eulerprompt}
\begin{euleroutput}
  2020-11-03,Tue,1'114.05
\end{euleroutput}
\begin{eulercomment}
Pertama-tama, kita dapat memecah baris tersebut menjadi token.
\end{eulercomment}
\begin{eulerprompt}
>vt=strtokens(line)
\end{eulerprompt}
\begin{euleroutput}
  2020-11-03
  Tue
  1'114.05
\end{euleroutput}
\begin{eulercomment}
Kemudian kita dapat mengevaluasi setiap elemen baris menggunakan
evaluasi yang sesuai.
\end{eulercomment}
\begin{eulerprompt}
>day(vt[1]),  ...
>indexof(["mon","tue","wed","thu","fri","sat","sun"],tolower(vt[2])),  ...
>strrepl(vt[3],"'","")()
\end{eulerprompt}
\begin{euleroutput}
  7.3816e+05
  2
  1114
\end{euleroutput}
\begin{eulercomment}
Dengan menggunakan ekspresi reguler, Anda dapat mengekstrak hampir
semua informasi dari baris data. 

Misalkan kita memiliki baris berikut dalam dokumen HTML.
\end{eulercomment}
\begin{eulerprompt}
>line="<tr><td>1145.45</td><td>5.6</td><td>-4.5</td><tr>"
\end{eulerprompt}
\begin{euleroutput}
  <tr><td>1145.45</td><td>5.6</td><td>-4.5</td><tr>
\end{euleroutput}
\begin{eulercomment}
Untuk mengekstrak ini, kita menggunakan ekspresi reguler yang mencari

\end{eulercomment}
\begin{eulerttcomment}
 - a closing bracket >,
 - any string not containing brackets with a sub-match "(...)",
 - an opening and a closing bracket using the shortest solution,
 - again any string not containing brackets,
 - and an opening bracket <.
\end{eulerttcomment}
\begin{eulercomment}

Ekspresi reguler agak sulit untuk dipelajari tetapi sangat kuat.
\end{eulercomment}
\begin{eulerprompt}
>\{pos,s,vt\}=strxfind(line,">([^<>]+)<.+?>([^<>]+)<");
\end{eulerprompt}
\begin{eulercomment}
Hasilnya adalah posisi kecocokan, string yang cocok, dan vektor string
untuk kecocokan sub.
\end{eulercomment}
\begin{eulerprompt}
>for k=1:length(vt); vt[k](), end;
\end{eulerprompt}
\begin{euleroutput}
  1145.5
  5.6
\end{euleroutput}
\begin{eulercomment}
Berikut adalah fungsi yang membaca semua item numerik antara \textless{}td\textgreater{} dan
\textless{}/td\textgreater{}.
\end{eulercomment}
\begin{eulerprompt}
>function readtd (line) ...
\end{eulerprompt}
\begin{eulerudf}
  v=[]; cp=0;
  repeat
     \{pos,s,vt\}=strxfind(line,"<td.*?>(.+?)</td>",cp);
     until pos==0;
     if length(vt)>0 then v=v|vt[1]; endif;
     cp=pos+strlen(s);
  end;
  return v;
  endfunction
\end{eulerudf}
\begin{eulerprompt}
>readtd(line+"<td>non-numerical</td>")
\end{eulerprompt}
\begin{euleroutput}
  1145.45
  5.6
  -4.5
  non-numerical
\end{euleroutput}
\eulerheading{Membaca dari Web }
\begin{eulercomment}
Sebuah situs web atau file dengan URL dapat dibuka di EMT dan dapat
dibaca baris demi baris. 

Dalam contoh ini, kita membaca versi terbaru dari situs EMT. Kami
menggunakan ekspresi reguler untuk mencari "Versi ..." dalam judul.
\end{eulercomment}
\begin{eulerprompt}
>function readversion () ...
\end{eulerprompt}
\begin{eulerudf}
  urlopen("http://www.euler-math-toolbox.de/Programs/Changes.html");
  repeat
    until urleof();
    s=urlgetline();
    k=strfind(s,"Version ",1);
    if k>0 then substring(s,k,strfind(s,"<",k)-1), break; endif;
  end;
  urlclose();
  endfunction
\end{eulerudf}
\begin{eulerprompt}
>readversion
\end{eulerprompt}
\begin{euleroutput}
  Version 2022-05-18
\end{euleroutput}
\eulerheading{Input dan Output Variabel }
\begin{eulercomment}
Anda dapat menulis variabel dalam bentuk definisi Euler ke dalam file
atau ke baris perintah.
\end{eulercomment}
\begin{eulerprompt}
>writevar(pi,"mypi");
\end{eulerprompt}
\begin{euleroutput}
  mypi = 3.141592653589793;
\end{euleroutput}
\begin{eulercomment}
Sebagai tes, kita menghasilkan file Euler di direktori kerja EMT.
\end{eulercomment}
\begin{eulerprompt}
>file="test.e"; ...
>writevar(random(2,2),"M",file); ...
>printfile(file,3)
\end{eulerprompt}
\begin{euleroutput}
  M = [ ..
  0.5991820585590205, 0.7960280262224293;
  0.5167243983231363, 0.2996684599070898];
\end{euleroutput}
\begin{eulercomment}
Kita sekarang dapat memuat file tersebut. Ini akan mendefinisikan
matriks M.
\end{eulercomment}
\begin{eulerprompt}
>load(file); show M,
\end{eulerprompt}
\begin{euleroutput}
  M = 
    0.59918   0.79603 
    0.51672   0.29967 
\end{euleroutput}
\begin{eulercomment}
Selain itu, jika writevar() digunakan pada variabel, itu akan mencetak
definisi variabel dengan nama variabel ini.
\end{eulercomment}
\begin{eulerprompt}
>writevar(M); writevar(inch$)
\end{eulerprompt}
\begin{euleroutput}
  M = [ ..
  0.5991820585590205, 0.7960280262224293;
  0.5167243983231363, 0.2996684599070898];
  inch$ = 0.0254;
\end{euleroutput}
\begin{eulercomment}
Kita juga dapat membuka file baru atau menambahkan ke file yang sudah
ada. Dalam contoh ini, kita menambahkan ke file yang sudah dibuat
sebelumnya.
\end{eulercomment}
\begin{eulerprompt}
>open(file,"a"); ...
>writevar(random(2,2),"M1"); ...
>writevar(random(3,1),"M2"); ...
>close();
>load(file); show M1; show M2;
\end{eulerprompt}
\begin{euleroutput}
  M1 = 
    0.30287   0.15372 
     0.7504   0.75401 
  M2 = 
    0.27213 
   0.053211 
    0.70249 
\end{euleroutput}
\begin{eulercomment}
Untuk menghapus file, gunakan fileremove().
\end{eulercomment}
\begin{eulerprompt}
>fileremove(file);
\end{eulerprompt}
\begin{eulercomment}
Sebuah vektor baris dalam file tidak memerlukan koma, jika setiap
angka ada di baris yang berbeda. Mari kita menghasilkan file seperti
itu, dengan menulis setiap baris satu per satu dengan writeln().
\end{eulercomment}
\begin{eulerprompt}
>open(file,"w"); writeln("M = ["); ...
>for i=1 to 5; writeln(""+random()); end; ...
>writeln("];"); close(); ...
>printfile(file)
\end{eulerprompt}
\begin{euleroutput}
  M = [
  0.344851384551
  0.0807510017715
  0.876519562911
  0.754157709472
  0.688392638934
  ];
\end{euleroutput}
\begin{eulerprompt}
>load(file); M
\end{eulerprompt}
\begin{euleroutput}
  [0.34485,  0.080751,  0.87652,  0.75416,  0.68839]
\end{euleroutput}
\eulersubheading{Contoh Soal}
\begin{eulercomment}
1. Diketahui data berat badan dari siswa kelas B\\
56, 55, 60, 47, 65, 56, 57, 62, 55, 40, 47, 45, 42, 68, 54, 48, 49,
38, 54, 47.

Carilah rata-rata dan standar deviasinya.
\end{eulercomment}
\begin{eulerprompt}
>TB=[56, 55, 60, 47, 65, 56, 57, 62, 55, 40, 47, 45, 42, 68, 54, 48, 49, 38, 54, 47]; ...
>mean(TB), dev(TB)
\end{eulerprompt}
\begin{euleroutput}
  52.25
  8.14264278837
\end{euleroutput}
\begin{eulercomment}
Jadi, rata-rata berat badan dari siswa kelas B adalah 52,25 dan
standar deviasinya 8,14264278837

2. Diketahui rata-rata nilai ulangan matematika kelas A adalah 78.5,
kelas B 79.7, kelas C 76.4, kelas D 72.7, kelas E 70.9, kelas F 71.2,
dan kelas G 80.2. Hitunglah rata-rata nilai matematika untuk seluruh
kelas.
\end{eulercomment}
\begin{eulerprompt}
>N=[78.5, 79.7, 76.4, 72.7, 70.9, 71.2, 80.2]; ...
>mean(N)
\end{eulerprompt}
\begin{euleroutput}
  75.6571428571
\end{euleroutput}
\begin{eulercomment}
Jadi, rata-rata nilai ulangan matematika dari kelas A sampai G adalah
75.65.


3. Data berikut menunjukkan nilai yang diperoleh 50 siswa kelas 10
pada ujian mata pelajaran biologi.\\
Siswa yang mendapat nilai dalam rentang 61-65 sebanyak 2 orang, dalam
rentang 66-70 sebanyak 3 orang, dalam rentang 71-75 sebanyak 9 orang,
dalam rentang 76-80 sebanyak 10 orang, dalam rentang 81-85 sebanyak 16
orang, dalam rentang 86-90 sebanyak 7 orang, dan dalam rentang 91-95
sebanyak 3 orang.\\
Tentukan rata-rata, median, dan histogram dari data nilai ujian
biologi kelas 10!

Penyelesaian :\\
Menentukan tepi bawah kelas yang terkecil
\end{eulercomment}
\begin{eulerprompt}
>61-0.5
\end{eulerprompt}
\begin{euleroutput}
  60.5
\end{euleroutput}
\begin{eulercomment}
Menentukan panjang kelas
\end{eulercomment}
\begin{eulerprompt}
>(65-61)+1
\end{eulerprompt}
\begin{euleroutput}
  5
\end{euleroutput}
\begin{eulercomment}
Menentukan tepi atas kelas yang terbesar
\end{eulercomment}
\begin{eulerprompt}
>95+0.5
\end{eulerprompt}
\begin{euleroutput}
  95.5
\end{euleroutput}
\begin{eulerprompt}
>r=60.5:5:95.5; v=[2,3,9,10,16,7,3];
>T:=r[1:7]' | r[2:8]' | v'; writetable(T,labc=["TB","TA","Frek"])
\end{eulerprompt}
\begin{euleroutput}
          TB        TA      Frek
        60.5      65.5         2
        65.5      70.5         3
        70.5      75.5         9
        75.5      80.5        10
        80.5      85.5        16
        85.5      90.5         7
        90.5      95.5         3
\end{euleroutput}
\begin{eulercomment}
Mencari Rata-rata

Menentukan nilai tengah
\end{eulercomment}
\begin{eulerprompt}
>(T[,1]+T[,2])/2
\end{eulerprompt}
\begin{euleroutput}
             63 
             68 
             73 
             78 
             83 
             88 
             93 
\end{euleroutput}
\begin{eulerprompt}
>t=fold(r,[0.5,0.5])
\end{eulerprompt}
\begin{euleroutput}
  [63,  68,  73,  78,  83,  88,  93]
\end{euleroutput}
\begin{eulerprompt}
>mean(t,v)
\end{eulerprompt}
\begin{euleroutput}
  79.8
\end{euleroutput}
\begin{eulercomment}
Mencari Median

Berdasarkan data, median berada pada urutan ke 25, maka median berada
pada kelas 80.5-85.5
\end{eulercomment}
\begin{eulerprompt}
>Tb=80.5, p=5, n=50, Fks=24, fm=16
\end{eulerprompt}
\begin{euleroutput}
  80.5
  5
  50
  24
\end{euleroutput}
\begin{eulerprompt}
>Tb+p*(1/2*n-Fks)/fm
\end{eulerprompt}
\begin{euleroutput}
  80.8125
\end{euleroutput}
\begin{eulercomment}
Jadi, rata-rata untuk nilai ujian biologi kelas 10 adalah 79.8 dan
mediannya adalah 80.81.

Histogram dari data nilai biologi kelas 10
\end{eulercomment}
\begin{eulerprompt}
>plot2d(r,v,a=61,b=95,c=0,d=20,bar=1,style="/"):
\end{eulerprompt}
\eulerimg{27}{images/22305141029_EMT4Statistika_Theresia Selvina-050.png}
\end{eulernotebook}
\end{document}
